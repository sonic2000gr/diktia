%
% section 5.1
%
\setcounter{section}{0}
\section{Εγκατεστημένο Τηλεφωνικό Δίκτυο}

Η κανονική τηλεφωνική εγκατάσταση αποτελείται από ένα ζευγάρι χάλκινων καλωδίων που εγκαθιστά  μια τηλεφωνική εταιρεία. Τα χάλκινα καλώδια που χρησιμοποιούνται στο τηλεφωνικό δίκτυο έχουν αρκετό εύρος ζώνης και μπορούν να μεταφέρουν αρκετά μεγαλύτερες συχνότητες από αυτές που χρησιμοποιούνται για τη μεταφορά φωνής. Στις συνδέσεις DSL αυτό το έξτρα εύρος ζώνης χρησιμοποιείται για να μεταφέρει πληροφορίες χωρίς να παρεμβάλλει τις επικοινωνίες φωνής που γίνονται ταυτόχρονα μέσα στην ίδια γραμμή.

Στην τηλεφωνική συνομιλία, οι συχνότητες που χρησιμοποιούνται είναι από 0 -- 3400 Hz. Οι συχνότητες αυτές έχουν επιλεχθεί καθώς εκεί βρίσκεται η περιοχή ομιλίας της ανθρώπινης φωνής. O περιορισμός των συχνοτήτων επιτρέπει επίσης στην τηλεφωνική εταιρεία να πακετάρει πολλά καλώδια σε μικρό χώρο χωρίς να ανησυχεί για παρεμβολές (crosstalk) μεταξύ τους. Η περιοχή αυτή είναι πολύ μικρή σε σχέση για παράδειγμα με τις περιοχές συχνοτήτων που αναπαράγονται από ένα στερεοφωνικό συγκρότημα (από 20 - 200000 Hz).  Τα τηλεφωνικά καλώδια έχουν τη δυνατότητα να μεταφέρουν και να χειρισθούν σήματα συχνότητας αρκετών εκατομμυρίων κύκλων (MHz).  Τα σύγχρονα μηχανήματα στέλνουν ψηφιακά και όχι αναλογικά δεδομένα και μπορούν να χρησιμοποιήσουν με ασφάλεια πολύ μεγαλύτερο εύρος ζώνης της τηλεφωνικής γραμμής χωρίς πρόβλημα παρεμβολών.