%
% section 6.1
%
\section{Σύστημα Ονοματολογίας DNS}
\label{sec:sec61}

Όπως είδαμε στην ενότητα \ref{sec:sec34} κάθε υπολογιστής που μετέχει σε ένα δίκτυο TCP/IP διαθέτει μια μοναδική διεύθυνση IP με την οποία μπορεί να αναγνωριστεί. Καθώς όμως οι άνθρωποι δεν είναι ιδιαίτερα καλοί στην απομνημόνευση αριθμών (και μάλιστα της μορφής που έχουν οι διευθύνσεις) προτιμούμε να αντιστοιχίζουμε ονόματα σε αυτές τις αριθμητικές διευθύνσεις. 

Στη ενότητα \ref{sec:sec34} είδαμε ότι ένας τρόπος αντιστοίχισης ονομάτων σε διευθύνσεις είναι μέσω του αρχείου hosts το οποίο υπάρχει μέχρι και σήμερα ουσιαστικά σε όλα τα λειτουργικά συστήματα. Στο αρχείο αυτό -- το οποίο είναι απλό κείμενο -- υπάρχουν γραμμές της μορφής 

\begin{center}
<διεύθυνση IP> \hspace{20mm} <όνομα υπολογιστή>
\end{center}

Για παράδειγμα, σε ένα μικρό δίκτυο ένα αρχείο hosts θα μπορούσε να περιέχει τα παρακάτω:

\ttfamily
\hspace{30mm}128.174.5.1 \hspace{20mm} atlas

\hspace{30mm}128.174.5.2 \hspace{20mm} kronos

\hspace{30mm}128.174.5.2 \hspace{20mm} aris
\normalfont

Το σύστημα αυτό προφανώς μπορεί να λειτουργήσει σε μικρά μόνο δίκτυα, καθώς έχει σοβαρά μειονεκτήματα:

\begin{itemize}
\item Πρέπει να υπάρχει αντίγραφο του αρχείου σε κάθε μηχάνημα του δικτύου
\item Όλα τα αντίγραφα πρέπει να ενημερώνονται κάθε φορά που γίνεται μια αλλαγή
\item Αν το αρχείο έχει πάρα πολλές καταχωρίσεις, η αναζήτηση σε αυτό θα είναι αργή
\end{itemize}

Είναι προφανές ότι αν μιλάμε για ονομασίες μηχανημάτων (και τοποθεσιών) στο Διαδίκτυο η λύση του αρχείου hosts είναι ανέφικτη:

\begin{itemize}
\item Το αρχείο hosts έχει επίπεδη μορφή: δεν διακρίνει υποχρεωτικά περιοχές ή υποδίκτυα. Οι υπολογιστές μπορεί να έχουν απλά ένα όνομα
\item θα ήταν αδύνατο να έχουμε ενημερωμένα αντίγραφα σε όλους τους υπολογιστές του Διαδικτύου αλλά\ldots
\item \ldots{}ακόμα και αν μπορούσαμε, το αρχείο θα ήταν τόσο μεγάλο που η αναζήτηση σε αυτό θα ήταν αδύνατη
\end{itemize}

Είναι φανερό ότι όσο αφορά το Internet χρειαζόμαστε ένα διαφορετικό τρόπο αντιμετώπισης του προβλήματος αντιστοίχισης ονομάτων σε διευθύνσεις και το αντίστροφο. Η λύση σε αυτό είναι το σύστημα DNS, που εκτελείται στο επίπεδο εφαρμογής και ακολουθεί το σύστημα πελάτη -- εξυπηρετητή.

Το σύστημα DNS προχωρά αρκετά παραπέρα από το αρχείο hosts:

\begin{itemize}
\item Επιτρέπει τον διαχωρισμό της ονοματολογίας σε περιοχές και υποπεριοχές αντί για την επίπεδη λογική του αρχείου hosts
\item Καταχωρεί τα δεδομένα του σε μια κατανεμημένη βάση δεδομένων για γρήγορη αναζήτηση
\item Κατανέμει το φορτίο εργασίας σε πάρα πολλούς υπολογιστές
\item Ένα μηχάνημα -- πελάτης (ο υπολογιστής π.χ. του τελικού χρήστη) δεν χρειάζεται να έχει καμιά διεύθυνση αποθηκευμένη, εκτός από το IP του υπολογιστή που θα απαντά στα ερωτήματα DNS (και το οποίο μπορεί επίσης να αποσταλεί μέσω της υπηρεσίας DHCP)
\end{itemize}

Πρέπει εδώ να τονίσουμε ότι το DNS λειτουργεί ως \emph{κατανεμημένη} βάση δεδομένων: δεν υπάρχει ένας και μοναδικός υπολογιστής που να διαθέτει όλες τις πληροφορίες της βάσης δεδομένων. Καθώς το DNS χωρίζει το Internet σε περιοχές και υποπεριοχές, διαφορετικοί εξυπηρετητές είναι υπεύθυνοι για κάθε μια από αυτές. Το σύστημα επιτρέπει ένα εξυπηρετητή να ρωτάει κάποιον άλλο προκειμένου να μάθει μια διεύθυνση που είναι έξω από την περιοχή ευθύνης του. Ένα σύστημα που δεν θα λειτουργούσε με κατανεμημένο τρόπο θα κατέρρεε πολύ γρήγορα καθώς δεν υπάρχει υπολογιστής και γραμμή δικτύου που θα μπορούσε να αντέξει τόσο τεράστια κίνηση από ερωτήματα.

Σε γενικές γραμμές το σύστημα DNS (Domain Name System ή Σύστημα Ονομασίας Περιοχών) είναι μια κατανεμημένη βάση δεδομένων που περιλαμβάνει:

\begin{itemize}
\item Το \emph{χώρο ονομάτων}
\item Τους \emph{εξυπηρετητές} μέσω των οποίων γίνεται διαθέσιμος ο χώρος ονομάτων
\item Τους \emph{αναλυτές (resolvers)} που θέτουν ερωτήματα στους εξυπηρετητές σχετικά με ονόματα και διευθύνσεις που περιέχονται στο χώρο ονομάτων
\end{itemize}

Από τα παραπάνω, ένα μηχάνημα -- πελάτης χρειάζεται να διαθέτει μόνο τον αναλυτή (resolver).

Το σύστημα DNS λειτουργεί στο επίπεδο εφαρμογής και εξυπηρετεί ουσιαστικά οποιαδήποτε συσκευή χρειάζεται να γνωρίζει IP διεύθυνση που αντιστοιχεί σε ένα όνομα (ή το αντίθετο), περιλαμβανομένων τελικών υπολογιστών (hosts), δρομολογητών (routers) αλλά και εξυπηρετητών DNS
μεταξύ τους. Το DNS είναι βασική υπηρεσία του Διαδικτύου και αναζητήσεις μπορεί να γίνονται από οποιοδήποτε μηχάνημα και υπηρεσία. Τυπικά οι αναλυτές στα μηχανήματα των χρηστών είναι ρυθμισμένοι να απευθύνουν ερωτήματα μόνο σε ένα ή δύο εξυπηρετητές DNS. Όταν ένας εξυπηρετητής DNS ερωτηθεί για ένα όνομα που δεν γνωρίζει (γιατί είναι έξω από τη ζώνη ευθύνης του), θα ρωτήσει άλλους εξυπηρετητές DNS προκειμένου να μάθει την απάντηση και να την επιστρέψει στον αναλυτή. Η απάντηση αυτή αποθηκεύεται προσωρινά προκειμένου να επιταχυνθεί η λειτουργία αν τεθεί ξανά η ίδια ερώτηση από άλλο μηχάνημα.