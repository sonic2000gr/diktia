%
% File for Section 8.1
%
\section{Βασικές Έννοιες Ασφάλειας Δεδομένων}

Για να κατανοήσουμε την έννοια της ασφάλειας, θα πρέπει:

\begin{itemize}
\item Να σκεφτούμε τι σημαίνει για τον καθένα μας. Η ασφάλεια αντιμετωπίζεται διαφορετικά από ιδιώτες και από εταιρίες
\item Να βρούμε ποιοι είναι αυτοί που προσπαθούν να την παραβιάσουν
\item Να ανακαλύψουμε τις προθέσεις των εισβολέων και το πιθανό όφελος τους από μια επιτυχή παραβίαση
\end{itemize}

Όταν σκεφτόμαστε την ασφάλεια σε προσωπικό επίπεδο και σε σχέση με την τεχνολογία, το Διαδίκτυο και τους υπολογιστές, συνήθως σκεφτόμαστε υποκλοπή δεδομένων όπως τον αριθμό της πιστωτικής μας κάρτας ή τους κωδικούς του e-banking με σκοπό προφανώς το οικονομικό όφελος. Άνθρωποι που έχουν δημόσιο προφίλ (ηθοποιοί, τραγουδιστές, πολιτικοί κλπ) ή που ασχολούνται με τα κοινά μπορεί να έχουν στις συσκευές τους ευαίσθητα δεδομένα που πρέπει να παραμείνουν μυστικά. Τυχόν διαρροή τέτοιων πληροφοριών μπορεί να είναι επιζήμια για το ίδιο το άτομο ή τον οργανισμό στον οποίο εργάζεται. Ο μέσος άνθρωπος πάντως σε γενικές γραμμές είναι αδιάφορος για την ασφάλεια πέρα από τις πληροφορίες που μεταφέρει και αποθηκεύει σε υπολογιστικά συστήματα και που έχουν να κάνουν με ηλεκτρονικές συναλλαγές (η υποκλοπή των οποίων μπορεί να προκαλέσει απώλεια χρημάτων).

Μπορούμε να θεωρήσουμε ότι ασφάλεια γενικότερα είναι η προσπάθεια προστασίας από εξωτερικές επιβουλές (κακόβουλες ενέργειες) στις πληροφορίες και τα συστήματα κατά τη λειτουργία τους ή κατά την επικοινωνία τους με άλλα συστήματα. 

Κάθε πράγμα που θέλουμε να προστατεύσουμε αποτελεί για μας ένα \emph{αγαθό} η απώλεια του οποίου μπορεί να μας προκαλέσει οικονομική ή άλλη ζημιά. 

\emph{Αγαθό ή πόρος} ενός υπολογιστικού / πληροφοριακούς συστήματος είναι κάθε αντικείμενο που ανήκει ή στηρίζει το σύστημα και το οποίο αξίζει να προστατευθεί.

Σε μια εταιρία, τα παραπάνω αγαθά μπορεί να ανήκουν τόσο στο υλικό όσο και στο λογισμικό του συστήματος και προφανώς περιέχουν επίσης τα δεδομένα (τα οποία πολλές φορές είναι αναντικατάστατα σε σχέση με όλα τα άλλα τμήματα):

Αγαθά είναι:

\begin{itemize}
\item Κτήρια, υπολογιστές, δικτυακός εξοπλισμός και υποδομή
\item Έπιπλα, γραφεία κλπ.
\item Αρχεία (ηλεκτρονικά και έντυπα), πληροφορίες σε συστήματα βάσεων δεδομένων κλπ.
\item Λογισμικό εφαρμογών, λειτουργικά συστήματα κλπ.
\end{itemize}

Για να είναι αποτελεσματική η προστασία μας στις επιθέσεις, θα πρέπει αρχικά να κατανοούμε ποιοι αποτελούν την απειλή, ποια δικά μας δεδομένα τους είναι χρήσιμα και πως μπορούν να τα χρησιμοποιήσουν εναντίον μας. Για παράδειγμα ένα πολύ σημαντικό όπλο στην ασφάλεια δεδομένων αποτελεί η διαδικασία της κρυπτογράφησης. Η πρώτη ισχυρή κρυπτογράφηση έγινε από τους Γερμανούς κατά το δεύτερο παγκόσμιο πόλεμο: έπρεπε να μεταδίδουν μηνύματα προς τα υποβρύχια τους μέσω ασυρμάτου χωρίς να μπορεί να γίνει υποκλοπή από τους συμμάχους. Για το σκοπό αυτό έφτιαξαν τη μηχανή Enigma. Ωστόσο στην Αγγλία, οι σύμμαχοι ανακάλυψαν αδυναμίες τόσο στη μηχανή όσο και στη διαδικασία μετάδοσης μηνυμάτων και κατάφερναν να αποκωδικοποιούν τα μηνύματα σε καθημερινή βάση.

Από τα παραπάνω είναι προφανές ότι οι κυβερνήσεις είναι αυτές που διαθέτουν τον καλύτερο εξοπλισμό τόσο για να προστατεύσουν τα δεδομένα τους όσο και για να προβούν σε παραβιάσεις ασφαλείας εναντίον άλλων κρατών. 

Μια εταιρεία ή οργανισμός με πολλούς εργαζόμενους και πελάτες συνήθως απευθύνεται σε συμβούλους ασφαλείας. Σε πολλές περιπτώσεις επιλέγεται μια λύση μεγάλου κόστους που προσπαθεί να δημιουργήσει τη μεγαλύτερη δυνατή ασφάλεια. Η πραγματικότητα όμως είναι ότι \emph{δεν μπορεί ποτέ να υπάρξει απόλυτη ασφάλεια}, εκτός αν δεν έχουμε ευαίσθητα δεδομένα και δεν υπάρχουν μυστικά που πρέπει να μεταδοθούν. Σε ένα υπολογιστικό σύστημα η ασφάλεια είναι πάντα τόσο καλή όσο είναι ο πιο αδύναμος κρίκος του, και αυτός αποδεικνύεται συχνά ότι είναι ο άνθρωπος. 

Επίσης η μεγαλύτερη ασφάλεια κάνει συνήθως ένα σύστημα πιο δύσχρηστο: οι χρήστες καλούνται συνέχεια να εισάγουν κωδικούς και οι κινήσεις τους ελέγχονται τόσο συχνά που το σύστημα μπορεί να είναι δυσκίνητο. Αν και αρχικός σκοπός είναι η αύξηση της ασφάλειας, ένα δύσχρηστο σύστημα στην πραγματικότητα μειώνει την ασφάλεια: σύντομα οι χρήστες θα ψάξουν να βρουν τρόπους για να παρακάμψουν ή να συντομεύσουν τις διαδικασίες αντικαθιστώντας τις με άλλες, ανασφαλείς.

Τελικά η ασφάλεια βασίζεται πάντα σε μια ανάλυση αντιστάθμισης του κόστους και των ωφελημάτων που μπορούν να επιτευχθούν. Είναι ένας συμβιβασμός που στη μια μεριά έχει το κόστος της απώλειας ή διαρροής των δεδομένων και στην άλλη το κόστος προφύλαξης τους από διαφορετικές απειλές. Σε αντίθεση με τους ειδικούς ασφαλείας που προτείνουν λύσεις με το μέγιστο κόστος, η πραγματική ασφάλεια συνήθως αναγνωρίζει τις συνηθισμένες απειλές και προσπαθεί να λάβει τα απαραίτητα μέτρα για αυτές, ενώ μπορεί να αγνοεί απειλές που είναι σπάνιες ή απίθανες.

Για παράδειγμα, μια συνηθισμένη επίθεση σε μια τράπεζα μπορεί να έχεις ως αποτέλεσμα την υποκλοπή στοιχείων πελατών όπως τραπεζικοί λογαριασμοί και πιστωτικές κάρτες. Το σύστημα ασφαλείας της τράπεζας θα πρέπει να είναι προετοιμασμένο για τέτοιες επιθέσεις. Μέχρι τι επίπεδο όμως; Τι απώλεια έχει η τράπεζα από τη διαρροή μιας πιστωτικής κάρτας; Εκτός από την προφανή άμεση οικονομική ζημιά, υπάρχει και η έμμεση: οι πελάτες δεν έχουν πλέον εμπιστοσύνη στη τράπεζα. Απώλεια πελατών σημαίνει επίσης μελλοντική οικονομική απώλεια. 

Γενικά είναι πιο εύκολο να υπολογίσουμε μια άμεση ζημιά  (π.χ. από την καταστροφή ενός υλικού) από την έμμεση (απώλεια εμπιστοσύνης). Ένα άλλο κόστος μπορεί να είναι οι νομικές συνέπειες που προκύπτουν ως αποτέλεσμα της παραβίασης ασφάλειας. Τέλος, αν μια εταιρεία βασίζεται στην αδιάλειπτη παροχή υπηρεσιών μέσω Διαδικτύου και δεχθεί επίθεση άρνησης υπηρεσίας (DOS attack) η απώλεια χρημάτων (πέρα από τις νομικές συνέπειες) μπορεί να είναι ανυπολόγιστη. Σκεφτείτε για παράδειγμα μια εταιρεία όπως το Amazon: αν το site του Amazon σταματήσει ξαφνικά να είναι προσβάσιμο, θα χάνονται αρκετά εκατομμύρια πωλήσεων κάθε λεπτό.


