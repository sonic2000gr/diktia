\chapter{Μεθοδολογία Ασκήσεων Υποδικτύωσης}
\newpage
\section{Μεθοδολογία Ασκήσεων Υποδικτύωσης}
\label{sec:appa}
Για να επιλύσουμε ασκήσεις υποδικτύωσης θα πρέπει:

\begin{itemize}
\item Να γνωρίζουμε μετατροπή από δυαδικό στο δεκαδικό και το ανάποδο (το βιβλίο και το βοήθημα περιγράφουν κάποιους εύκολους τρόπους).
\item Να γνωρίζουμε τις δυνάμεις του δύο (όχι απαραίτητα απ’έξω βέβαια, αρκεί να γράψουμε το αντίστοιχο πινακάκι πριν ξεκινήσουμε).
\end{itemize}

Τα παραπάνω είναι απαραίτητα, καθώς για να δουλέψουμε με τις μάσκες στην υποδικτύωση θα πρέπει να έχουμε τις αντίστοιχες οκτάδες στο δυαδικό. Τα δεδομένα / ζητούμενα της άσκησης μπορεί να δίνονται / ζητούνται σε οποιοδήποτε από τα δύο συστήματα. Καλό θα είναι να εξασκηθείτε στις μετατροπές. Επίσης συνηθίστε να ελέγχετε το αποτέλεσμα μιας μετατροπής κάνοντας την αντίστροφα.

\begin{inthebox}
\textbf{Χρήσιμο tip:} Ένας αριθμός στο δυαδικό με το τελευταίο ψηφίο 0 είναι ζυγός, ενώ με 1 είναι μονός. Είναι η πιο γρήγορη αρχική επαλήθευση που μπορείτε να κάνετε.\\
\end{inthebox}

\section*{Μάσκα Δικτύου και Διευθύνσεις Δικτύου / Εκπομπής}

Η μάσκα δικτύου σε μια άσκηση μπορεί να δίνεται σε οποιαδήποτε από τις παρακάτω μορφές:

\begin{verbatim}
Δεκαδική: 255.255.240.0
Δυαδικό: 11111111	11111111	11110000	00000000
CIDR (πρόθεμα): /20
\end{verbatim}

Όταν θέλουμε να εργαστούμε με τη μάσκα για να βρούμε διευθύνσεις δικτύου, εκπομπής ή να κάνουμε υποδικτύωση, πάντα θα πρέπει να τη φέρουμε στη δυαδική της μορφή.

\subsection*{Παράδειγμα 1}

Δίνεται η διεύθυνση IP 192.168.3.124 με μάσκα 255.255.255.0. Να υπολογιστεί η Διεύθυνση Δικτύου και η Διεύθυνση Εκπομπής.

\subsubsection*{Απάντηση}

Θα πρέπει να γράψουμε τη μάσκα και τη διεύθυνση IP στις αντίστοιχες δυαδικές μορφές κάνοντας τη μετατροπή. Για τη μάσκα είναι εύκολο να θυμάστε φυσικά ότι το 255 (που συναντάται πολύ συχνά) είναι απλά οκτώ άσοι: 11111111. Φτιάχνουμε το παρακάτω πινακάκι:

\begin{center}
\fontsize{9}{11}
\ttfamily
\begin{tabular}{|c|c|c|c|c|}
\hline
 IP (Δεκαδικό): & 192. & 168. & 3. & 124 \\ 
\hline
 IP (Δυαδικό): & 1100 0000 & 1010 1000 & 0000 0011 & 0111 1100\\
\hline
 Μάσκα (Δυαδικό): & 1111 1111 & 1111 1111 & 1111 1111 & 0000 0000\\
 \hline
 Διεύθυνση Δικτύου: & 1100 0000 & 1010 1000 & 0000 0011 & 0000 0000\\
 \hline
 Διεύθ. Δικτύου (Δεκαδικό): & 192. & 168. & 3. & 0\\
 \hline 
\end{tabular}
\normalfont
\end{center}

Βοηθάει αν γράφουμε τους δυαδικούς χωρισμένους σε τετράδες ψηφίων ώστε να μη μπερδευόμαστε στο μέτρημα. Δεν είναι ωστόσο απαραίτητο.

Η διεύθυνση δικτύου προκύπτει από το λογικό ``ΚΑΙ'' της μάσκας και της διεύθυνσης IP.

\begin{inthebox}
\textbf{Χρήσιμο tip:} Όπου η μάσκα είναι 255 (ή 11111111), προκύπτει ακριβώς ο ίδιος αριθμός που αναγράφεται στην αντίστοιχη οκτάδα της διεύθυνσης IP. Όπου η μάσκα είναι μηδέν (ή 00000000) προκύπτει μηδέν στην αντίστοιχη οκτάδα. \textbf{Προσέξτε στις μάσκες δικτύου που έχουν άλλους αριθμούς: θα πρέπει να το κάνετε ανά ψηφίο.}\\
\end{inthebox}

Για να υπολογίσουμε τη διεύθυνση εκπομπής, θα πρέπει να πάρουμε τη \textbf{διεύθυνση δικτύου που βρήκαμε πριν} και να βάλουμε 1 (ένα) σε όλα τα bit που ανήκουν στο τμήμα του υπολογιστή. Οπότε είναι σκόπιμο να γράψετε σε ένα πίνακα τη διεύθυνση δικτύου και τη μάσκα ξανά:

\begin{center}
\fontsize{9}{11}
\ttfamily
\begin{tabular}{|c|c|c|c|c|}
\hline
 Διεύθυνση Δικτύου: & 1100 0000 & 1010 1000 & 0000 0011 & 0000 0000\\ 
\hline
Μάσκα (Δυαδικό): & 1111 1111 & 1111 1111 & 1111 1111 & \textcolor{red}{0000 0000} \\
\hline
Διεύθυνση Εκπομπής: & 1100 0000 & 1010 1000 & 0000 0011 & \textcolor{blue}{1111 1111}\\
 \hline
 Διεύθ. Εκπομπής (Δεκαδικό): & 192. & 168. & 3. & 255\\
 \hline
\end{tabular}
\normalfont
\end{center}

Κάνουμε ``1'' όλα τα bit στη διεύθυνση δικτύου στα οποία τα αντίστοιχα ψηφία της μάσκας είναι μηδέν. Έπειτα μετατρέπουμε ξανά στο δεκαδικό και παίρνουμε τη διεύθυνση εκπομπής 192.168.3.255. Προσοχή, για να βρούμε τη διεύθυνση εκπομπής πρέπει να ξεκινήσουμε από τη \textbf{διεύθυνση δικτύου} και όχι την IP!

Όπως καταλαβαίνετε, η εύρεση της διεύθυνσης εκπομπής είναι πολύ εύκολη αν έχουμε μάσκα με τιμές μόνο 255 και 0. Αν ωστόσο η μάσκα που έχουμε αντιστοιχεί σε υποδικτύωση (ή υπερδικτύωση) θα πρέπει να κάνετε το παραπάνω προσεκτικά. Δείτε το παρακάτω παράδειγμα.

\subsection*{Παράδειγμα 2}

Δίνεται η διεύθυνση IP 192.168.5.73/27. Να βρείτε τη διεύθυνση δικτύου και τη διεύθυνση εκπομπής.

\subsubsection*{Απάντηση}

Στο συγκεκριμένο παράδειγμα μας δίνεται η μάσκα σε μορφή CIDR.  Οπότε ξέρουμε ότι απλά θα γράψουμε 27 άσους. Μια τέτοια μάσκα δεν αντιστοιχεί σε μια τυποποιημένη κλάση (A,B,C). Έχουμε δώσει τρία παραπάνω bit στο τμήμα δικτύου και έτσι το τμήμα υπολογιστή διαθέτει μόνο 5 bit. Πρόκειται δηλ. για υποδικτύωση. Κάνουμε ξανά τον αντίστοιχο πίνακα:

\begin{center}
\fontsize{9}{11}
\ttfamily
\begin{tabular}{|c|c|c|c|c|}
\hline
 IP (Δεκαδικό): & 192. & 168. & 5. & 73 \\ 
\hline
 IP (Δυαδικό): & 1100 0000 & 1010 1000 & 0000 0101 & 0100 1001\\
\hline
 Μάσκα (Δυαδικό): & 1111 1111 & 1111 1111 & 1111 1111 & \textcolor{red}{1110} 0000\\
 \hline
 Διεύθυνση Δικτύου: & 1100 0000 & 1010 1000 & 0000 0101 & \textcolor{red}{0100} 0000\\
 \hline
 Διεύθ. Δικτύου (Δεκαδικό): & 192. & 168. & 5. & 64\\
 \hline 
\end{tabular}
\normalfont
\end{center}

Δώστε προσοχή στην τελευταία οκτάδα!

Αντίστοιχα, (και με την ίδια προσοχή!) θα πρέπει να υπολογίσουμε τη διεύθυνση εκπομπής. Ξεκινάμε από τη \textbf{διεύθυνση δικτύου} που βρήκαμε πριν και με τη μάσκα που έχουμε:

\begin{center}
\fontsize{9}{11}
\ttfamily
\begin{tabular}{|c|c|c|c|c|}
\hline
 Διεύθυνση Δικτύου: & 1100 0000 & 1010 1000 & 0000 0101 & 0100 0000\\ 
\hline
Μάσκα (Δυαδικό): & 1111 1111 & 1111 1111 & 1111 1111 & 111\textcolor{red}{0 0000} \\
\hline
Διεύθυνση Εκπομπής: & 1100 0000 & 1010 1000 & 0000 0101 & 010\textcolor{blue}{1 1111}\\
 \hline
 Διεύθ. Εκπομπής (Δεκαδικό): & 192. & 168. & 5. & 95\\
 \hline
\end{tabular}
\normalfont
\end{center}

Η διεύθυνση εκπομπής προκύπτει όταν κάνουμε ``1'' τα ψηφία στη διεύθυνση δικτύου στα οποία τα αντίστοιχα ψηφία της μάσκας είναι ``0''. Δηλ. κάνουμε ``1'' τα ψηφία που ανήκουν στο τμήμα υπολογιστή. Και βλέπετε ότι σε αυτή τη περίπτωση η απάντηση δεν είναι προφανής (όπως όταν έχουμε μόνο 255 και 0 στη μάσκα).

Για να επαληθεύετε τα αποτελέσματα σας μπορείτε πάντα να επισκεφθείτε μια από τις πολλές σελίδες στο διαδίκτυο που υπολογίζουν τις αντίστοιχες διευθύνσεις. Π.χ. για διευθύνσεις δικτύου και εκπομπής, δείτε:

\begin{center}
\url{http://www.remotemonitoringsystems.ca/broadcast.php}
\end{center}

\subsection*{Ασκήσεις προς επίλυση}

\begin{enumerate}
\item Να βρείτε τη διεύθυνση δικτύου και εκπομπής σε ένα δίκτυο όπου μια διεύθυνση IP είναι 10.14.28.55 και η μάσκα είναι 255.240.0.0. (Τα αποτελέσματα να εκφραστούν και στο δεκαδικό σύστημα).
\item Να βρείτε τη διεύθυνση δικτύου και εκπομπής σε ένα δίκτυο με\\ IP 192.168.3.94 /26. (Τα αποτελέσματα να εκφραστούν και στο δεκαδικό σύστημα).
\item Να βρείτε τη διεύθυνση δικτύου και εκπομπής σε ένα δίκτυο με\\ IP 192.168.230.20 και μάσκα 255.255.248.0. (Τα αποτελέσματα να εκφραστούν και στο δεκαδικό σύστημα).
\end{enumerate}

\section*{Υποδικτύωση}

Στην υποδικτύωση, δίνουμε κάποια ψηφία από το τμήμα υπολογιστή στο τμήμα δικτύου. Έτσι για παράδειγμα, ενώ στην κλάση C έχουμε 24 bit στο τμήμα δικτύου και 8 στο τμήμα υπολογιστή, με την υποδικτύωση μπορούμε να μειώσουμε το τμήμα υπολογιστή και να αυξήσουμε το τμήμα δικτύου. Για παράδειγμα, αν δώσουμε 27 bit στο τμήμα δικτύου (με τη βοήθεια πάντα της μάσκας) θα μας μείνουν μόνο 5 bit στο τμήμα υπολογιστή. 

Χωρίζουμε ένα δίκτυο κλάσης C συνήθως για διαχειριστικούς λόγους: Δεν θέλουμε ένα δίκτυο με 254 μηχανήματα αλλά μερικά δίκτυα με λιγότερα (ένα για το λογιστήριο, ένα για την αποθήκη, ένα για τη μισθοδοσία κλπ). Χωρίζουμε ένα δίκτυο κλάσης B σε μικρότερα γιατί σχεδόν καμιά εταιρεία δεν θα χρησιμοποιήσει σε μια εγκατάσταση 65534 υπολογιστές: χωρίζοντας το σε μερικά κομμάτια π.χ. των 8000 υπολογιστών, μπορούμε να τα διαθέσουμε σε πολλές εταιρείες και να αποφύγουμε τη σπατάλη διευθύνσεων.

\begin{inthebox}
\textbf{Χρήσιμο tip:} Όταν δίνουμε bits από το τμήμα υπολογιστή στο τμήμα δικτύου, έχουμε υποδικτύωση. Όταν δίνουμε bits από το τμήμα δικτύου στο τμήμα υπολογιστή έχουμε υπερδικτύωση.\\
\end{inthebox}

\subsection*{Παραδείγματα Υποδικτύωσης}
\subsection*{Παράδειγμα 1}

Δίνεται η διεύθυνση δικτύου 192.168.12.0/24.

\begin{enumerate}
\item Να χωριστεί το δίκτυο σε 5 τουλάχιστον υποδίκτυα, να δοθούν οι διευθύνσεις δικτύου και εκπομπής για κάθε υποδίκτυο
\item Πόσους υπολογιστές έχει το κάθε υποδίκτυο;
\end{enumerate}

\subsubsection*{Απάντηση}

Είναι εμφανές ότι έχουμε ένα δίκτυο κλάσης C με μάσκα 255.255.255.0. Αν το χωρίσουμε σε 5 υποδίκτυα, το κάνουμε μάλλον για διαχειριστικούς λόγους.

Θα πρέπει να πάρουμε κάποια bit από το τμήμα υπολογιστή και να τα δώσουμε στο τμήμα δικτύου. Αλλά πόσα;

Αντί για ένα δίκτυο, θέλουμε πλέον 5. Με 2 bit επιπλέον μπορούμε να φτιάξουμε 2\textsuperscript{2}=4 δίκτυα ενώ με 3 bit, 2\textsuperscript{3}=8. Προφανώς τα δύο bit είναι λίγα, ενώ τα τρία περισσεύουν. Ωστόσο δεν έχουμε ενδιάμεση επιλογή και θα χρησιμοποιήσουμε τρία bit. Άλλωστε για αυτό το λόγο και η άσκηση λέει \textbf{τουλάχιστον} 5 υποδίκτυα, και όχι ακριβώς 5! Αν μας δώσουν πλήθος υποδικτύων που είναι δύναμη του 2, θα μπορέσουμε να το κάνουμε ακριβώς.

Στο σημείο αυτό είναι χρήσιμο να έχουμε το παρακάτω πινακάκι δυνάμεων του 2. Αν δεν είστε εξοικειωμένοι με τις δυνάμεις του 2 τουλάχιστον μέχρι το 2\textsuperscript{8} καλό θα είναι να το γράψετε πριν ξεκινήσετε την άσκηση για να το έχετε ως αναφορά:

\begin{center}
\begin{tabular}{|c|c|}
\hline
 \textbf{Ψηφία n} & \textbf{Πλήθος 2\textsuperscript{n}} \\ 
\hline
 1 & 2\\
\hline
 2 & 4\\
 \hline
 3 & 8\\
 \hline
 4 & 16\\
 \hline 
 5 & 32\\
 \hline
 6 & 64\\
 \hline
 7 & 128\\
 \hline
 8 & 256\\
 \hline
\end{tabular}
\end{center}

Και από τον πίνακα είναι εμφανές ότι για 5 υποδίκτυα χρειαζόμαστε 3 bit. Αυτά τα τρία bit θα πάρουν τιμή ``1'' στη μάσκα του δικτύου που θα φτιάξουμε!

Για να ξεκινήσουμε πρέπει να γράψουμε τη διεύθυνση δικτύου στο δυαδικό:

\begin{center}
\fontsize{9}{11}
\ttfamily
\begin{tabular}{|c|c|c|c|c|}
\hline
 Διεύθυνση Δικτύου: & 192. & 168. & 12. & 0 \\ 
\hline
 Δ. Δικτύου (Δυαδικό): & 1100 0000 & 1010 1000 & 0000 1100 & 0000 0000\\
\hline
 Μάσκα (Δυαδικό): & 1111 1111 & 1111 1111 & 1111 1111 & \textcolor{red}{1110} 0000\\
 \hline
 Μάσκα (Δεκαδικό): & 255. & 255. & 255 & 224\\
 \hline 
\end{tabular}
\normalfont
\end{center}

Δώσαμε τρία επιπλέον ψηφία από το τμήμα υπολογιστή στο τμήμα δικτύου, έτσι η νέα μάσκα είναι 255.255.255.224.

Μπορούμε τώρα να γράψουμε τα οκτώ υποδίκτυα που προκύπτουν (θυμηθείτε ότι περισσεύουν\ldots)

\begin{center}
\fontsize{10}{12}
\ttfamily
\begin{tabular}{|c|c|c|c|c|c|c|}
\hline
                \textbf{Α/Α}  & \textbf{1η οκτάδα}  & \textbf{2η οκτάδα}  & \textbf{3η οκτάδα}  & \multicolumn{2}{c|}{\textbf{4η οκτάδα }}  & \textbf{Διευθύνσεις} \\ \hline

\multirow{2}{*}{0} &\multirow{2}{*}{1100 0000} & \multirow{2}{*}{1010 1000} & \multirow{2}{*}{0000 1100} & \multirow{2}{*}{000}  & 00000 & 192.168.12.0 \\  \cline{6-7} 
                  & &  &  &                    & 11111 & 192.168.12.31  \\ \hline

\multirow{2}{*}{1} & \multirow{2}{*}{1100 0000}  & \multirow{2}{*}{1010 1000} & \multirow{2}{*}{0000 1100}  & \multirow{2}{*}{001}  & 00000  & 192.168.12.32 \\  \cline{6-7} 
                  &  &  &  & & 11111 & 192.168.12.63 \\ \hline
\multirow{2}{*}{2} & \multirow{2}{*}{1100 0000}  & \multirow{2}{*}{1010 1000}  & \multirow{2}{*}{0000 1100} & \multirow{2}{*}{010}  & 00000 & 192.168.12.64 \\  \cline{6-7} 
                  &  &  &  &    & 11111 & 192.168.12.95 \\ \hline
\multirow{2}{*}{3} & \multirow{2}{*}{1100 0000}  & \multirow{2}{*}{1010 1000} & \multirow{2}{*}{0000 1100}  & \multirow{2}{*}{011}  & 00000 & 192.168.12.96 \\  
\cline{6-7} 
                  &  &   &  & & 11111 & 192.168.12.127 \\ \hline
\multirow{2}{*}{4} & \multirow{2}{*}{1100 0000} & \multirow{2}{*}{1010 1000}  & \multirow{2}{*}{0000 1100} & \multirow{2}{*}{100}  & 00000 & 192.168.12.128 \\  \cline{6-7} 
                  &  &  &   & & 11111 & 192.168.12.159 \\ \hline
\multirow{2}{*}{5} & \multirow{2}{*}{1100 0000}  & \multirow{2}{*}{1010 1000}  & \multirow{2}{*}{0000 1100} & \multirow{2}{*}{101}  & 00000 &  192.168.12.160 \\  
\cline{6-7} 
                  & & &  &  & 11111 &  192.168.12.191 \\ \hline
\multirow{2}{*}{6} & \multirow{2}{*}{1100 0000}  & \multirow{2}{*}{1010 1000 } & \multirow{2}{*}{0000 1100} & \multirow{2}{*}{110}  & 00000 & 192.168.12.192 \\  \cline{6-7} 
                  &  & &  &  & 11111 & 192.168.12.223 \\ \hline
\multirow{2}{*}{7} & \multirow{2}{*}{1100 0000}  & \multirow{2}{*}{1010 1000} & \multirow{2}{*}{0000 1100}  & \multirow{2}{*}{111}  & 00000  &  192.168.12.224 \\  \cline{6-7} 
                  &  &  & & & 11111 & 192.168.12.255 \\ \hline
\end{tabular}
\normalfont
\end{center}

Πως τα γράψαμε; Θυμηθείτε στις τρεις πρώτες οκτάδες δεν υπάρχει καμιά αλλαγή: ανήκουν εξ’ολοκλήρου στο δίκτυο. Στη τέταρτη οκτάδα ωστόσο, τα τρία πρώτα bit δείχνουν το δίκτυο και τα άλλα πέντε τον υπολογιστή. Οπότε για κάθε ένα από τους 8 συνδυασμούς των τριών πρώτων ψηφίων τα άλλα πέντε μπορούν να πάρουν όλες τις τιμές από 00000 μέχρι 11111.

\textbf{Σε όλα αυτά τα υποδίκτυα, η πρώτη διεύθυνση που βρίσκουμε είναι η διεύθυνση δικτύου και η τελευταία η διεύθυνση εκπομπής!} Μπορείτε αν θέλετε να το επαληθεύσετε με τη βοήθεια της μάσκας υποδικτύου.

Έτσι μπορούμε να δώσουμε τον παρακάτω πίνακα απαντήσεων:

\begin{center}
\fontsize{10}{12}
\ttfamily
\begin{tabular}{|c|c|c|c|c|c|c|}
\hline
 \multirow{2}{*}{\textbf{A/A}} & \multirow{2}{*}{\textbf{Διεύθυνση}} & \multirow{2}{*}{\textbf{Διεύθυνση}} &  \multicolumn{2}{l|}{\multirow{2}{*}{\textbf{IP Από -- Εώς}}} & \multirow{2}{*}{\textbf{Πλήθος}} \\ 
& \textbf{Δικτύου} & \textbf{Εκπομπής} & \multicolumn{2}{l|}{} & \textbf{Υπολογιστών}\\
\hline
\multirow{2}{*}{0} & \multirow{2}{*}{192.168.12.0} & \multirow{2}{*}{192.168.12.31} & \multicolumn{2}{l|}{192.168.12.1} & \multirow{2}{*}{30} \\ \cline{4-5}
                  &                   &                   & \multicolumn{2}{l|}{192.168.12.30} &                   \\ \hline
\multirow{2}{*}{1} & \multirow{2}{*}{192.168.12.32} & \multirow{2}{*}{192.168.12.63} & \multicolumn{2}{l|}{192.168.12.33} & \multirow{2}{*}{30} \\ \cline{4-5}
                  &                   &                   & \multicolumn{2}{l|}{192.168.12.62} &                   \\ \hline
\multirow{2}{*}{2} & \multirow{2}{*}{192.168.12.64} & \multirow{2}{*}{192.168.12.95} & \multicolumn{2}{l|}{192.168.12.65} & \multirow{2}{*}{30} \\ \cline{4-5}
                  &                   &                   & \multicolumn{2}{l|}{192.168.12.94} &                   \\ \hline
\multirow{2}{*}{3} & \multirow{2}{*}{192.168.12.96} & \multirow{2}{*}{192.168.12.127} & \multicolumn{2}{l|}{192.168.12.97} & \multirow{2}{*}{30} \\ \cline{4-5}
                  &                   &                   & \multicolumn{2}{l|}{192.168.12.126} &                   \\ \hline
\multirow{2}{*}{4} & \multirow{2}{*}{192.168.12.128} & \multirow{2}{*}{192.168.12.159} & \multicolumn{2}{l|}{192.168.12.129} & \multirow{2}{*}{30} \\ \cline{4-5}
                  &                   &                   & \multicolumn{2}{l|}{192.168.12.158} &                   \\ \hline
\multirow{2}{*}{5} & \multirow{2}{*}{192.168.12.160} & \multirow{2}{*}{192.168.12.191} & \multicolumn{2}{l|}{192.168.12.161} & \multirow{2}{*}{30} \\ \cline{4-5}
                  &                   &                   & \multicolumn{2}{l|}{192.168.12.190} &                   \\ \hline
\multirow{2}{*}{6} & \multirow{2}{*}{192.168.12.192} & \multirow{2}{*}{192.168.12.223} & \multicolumn{2}{l|}{192.168.12.193} & \multirow{2}{*}{30} \\ \cline{4-5}
                  &                   &                   & \multicolumn{2}{l|}{192.168.12.222} &                   \\ \hline
\multirow{2}{*}{7} & \multirow{2}{*}{192.168.12.224} & \multirow{2}{*}{192.168.12.255} & \multicolumn{2}{l|}{192.168.12.225} & \multirow{2}{*}{30} \\ \cline{4-5}
                  &                   &                   & \multicolumn{2}{l|}{192.168.12.254} &                   \\ \hline
\end{tabular}
\normalfont
\end{center}

\subsection*{Παράδειγμα 2}

Ενώ στο πρώτο παράδειγμα μας ζήτησαν συγκεκριμένο αριθμό δικτύων, σε άλλη περίπτωση μπορεί να μας ζητήσουν να φτιάξουμε υποδίκτυα με συγκεκριμένο αριθμό μηχανημάτων. Για παράδειγμα:

Δίνεται η διεύθυνση δικτύου 192.168.14.0 με μάσκα 255.255.255.0 (δηλ /24). Να χωριστεί σε υποδίκτυα ώστε το καθένα από αυτά να έχει τουλάχιστον 14 μηχανήματα. 

\subsubsection*{Απάντηση}

Σκεφτόμαστε με τον ίδιο τρόπο όπως προηγουμένως, μόνο που τώρα υπολογίζουμε πόσα bit χρειαζόμαστε για τα μηχανήματα. Τα υπόλοιπα bit θα τα διαθέσουμε στο τμήμα δικτύου.

Για 14 μηχανήματα, χρειαζόμαστε 4 ψηφία, γιατί 2\textsuperscript{4}=16. Τα 3 ψηφία δεν αρκούν (2\textsuperscript{3}=8). Παρατηρήστε ότι με 4 ψηφία θα έχουμε \textbf{ακριβώς 14 μηχανήματα, γιατί χάνουμε δύο διευθύνσεις ανά υποδίκτυο (δικτύου και εκπομπής).}
 
Εδώ λοιπόν θα κρατήσουμε τα 4 τελευταία ψηφία της 4ης οκτάδας για το τμήμα υπολογιστή και θα δώσουμε τα άλλα 4 στο τμήμα υπολογιστή. 

Θα έχουμε λοιπόν συνολικά 16 υποδίκτυα, με 14 μηχανήματα στο καθένα.
Θα υπολογίσουμε αρχικά τη μάσκα δικτύου:

\begin{center}
\fontsize{9}{11}
\ttfamily
\begin{tabular}{|c|c|c|c|c|}
\hline
 Διεύθυνση Δικτύου: & 192. & 168. & 14. & 0 \\ 
\hline
 Δ. Δικτύου (Δυαδικό): & 1100 0000 & 1010 1000 & 0000 1110 & 0000 0000\\
\hline
 Μάσκα (Δυαδικό): & 1111 1111 & 1111 1111 & 1111 1111 & \textcolor{red}{1111} 0000\\
 \hline
 Μάσκα (Δεκαδικό): & 255. & 255. & 255 & 240\\
 \hline 
\end{tabular}
\normalfont
\end{center}

\begin{center}
\fontsize{10}{12}
\ttfamily
\begin{tabular}{|c|c|c|c|c|c|c|}
\hline
                \textbf{Α/Α}  & \textbf{1η οκτάδα}  & \textbf{2η οκτάδα}  & \textbf{3η οκτάδα}  & \multicolumn{2}{c|}{\textbf{4η οκτάδα }}  & \textbf{Διευθύνσεις} \\ \hline

\multirow{2}{*}{0} &\multirow{2}{*}{1100 0000} & \multirow{2}{*}{1010 1000} & \multirow{2}{*}{0000 1110} & \multirow{2}{*}{0000}  & 0000 & 192.168.14.0 \\  \cline{6-7} 
                  & &  &  &                    & 1111 & 192.168.14.15  \\ \hline

\multirow{2}{*}{1} & \multirow{2}{*}{1100 0000}  & \multirow{2}{*}{1010 1000} & \multirow{2}{*}{0000 1110}  & \multirow{2}{*}{0001}  & 0000  & 192.168.14.16 \\  \cline{6-7} 
                  &  &  &  & & 1111 & 192.168.14.31 \\ \hline
\multirow{2}{*}{2} & \multirow{2}{*}{1100 0000}  & \multirow{2}{*}{1010 1000}  & \multirow{2}{*}{0000 1110} & \multirow{2}{*}{0010}  & 0000 & 192.168.14.32 \\  \cline{6-7} 
                  &  &  &  &    & 1111 & 192.168.14.47 \\ \hline
\multirow{2}{*}{3} & \multirow{2}{*}{1100 0000}  & \multirow{2}{*}{1010 1000} & \multirow{2}{*}{0000 1110}  & \multirow{2}{*}{0011}  & 0000 & 192.168.14.48 \\  
\cline{6-7} 
                  &  &   &  & & 1111 & 192.168.14.63 \\ \hline
\multirow{2}{*}{4} & \multirow{2}{*}{1100 0000} & \multirow{2}{*}{1010 1000}  & \multirow{2}{*}{0000 1110} & \multirow{2}{*}{0100}  & 0000 & 192.168.14.64 \\  \cline{6-7} 
                  &  &  &   & & 1111 & 192.168.14.79 \\ \hline
\multirow{2}{*}{5} & \multirow{2}{*}{1100 0000}  & \multirow{2}{*}{1010 1000}  & \multirow{2}{*}{0000 1110} & \multirow{2}{*}{0101}  & 0000 &  192.168.14.80 \\  
\cline{6-7} 
                  & & &  &  & 1111 &  192.168.14.95 \\ \hline
\multirow{2}{*}{6} & \multirow{2}{*}{1100 0000} &  \multirow{2}{*}{1010 1000} & \multirow{2}{*}{0000 1110} & \multirow{2}{*}{0110}  & 0000 & 192.168.14.96 \\  \cline{6-7} 
                  &  & &  &  & 1111 & 192.168.12.111 \\ \hline
\multirow{2}{*}{7} & \multirow{2}{*}{1100 0000}  & \multirow{2}{*}{1010 1000} & \multirow{2}{*}{0000 1110}  & \multirow{2}{*}{0111}  & 0000  &  192.168.14.112 \\  \cline{6-7} 
                  &  &  & & & 1111 & 192.168.14.127 \\ \hline
\end{tabular}
\normalfont
\end{center}

Και ακόμα 8 υποδίκτυα που δεν δείχνουμε για οικονομία χώρου (και χαρτιού)!

Όπως καταλαβαίνετε, σε καθένα από αυτά τα υποδίκτυα η πρώτη διεύθυνση είναι η \textbf{διεύθυνση δικτύου και η τελευταία η διεύθυνση εκπομπής.} Το κάθε υποδίκτυο συνδέει ακριβώς 14 υπολογιστές. Συνολικά έχουμε 16 Χ 14 = 224 υπολογιστές αντί για 254.

Μπορείτε να δείτε πάντα σε ένα αντίστοιχο web calculator αν έχετε κάνει τους σωστούς υπολογισμούς:

\begin{center}
\url{http://jodies.de/ipcalc} 
\end{center}

Προσπαθήστε τώρα να λύσετε τη δραστηριότητα 3η του βιβλίου (σελ. 81) χωρίς τη βοήθεια του site :D 