%
% section 4.1
%
\section{Πρωτόκολλα Προσανατολισμένα στη Σύνδεση -- Χωρίς Σύνδεση}

Οι δικτυακές εφαρμογές που είναι εγκαταστημένες στους κόμβους ενός δικτύου (υπολογιστές, έξυπνες συσκευές, smartphones κλπ) επικοινωνούν ανταλλάσσοντας μεταξύ τους μηνύματα δεδομένων. Το επίπεδο μεταφοράς παρέχει τις διαδικασίες που αναλαμβάνουν τη μεταφορά μηνυμάτων με διαφανή τρόπο από τις εφαρμογές που τα παράγουν.

Το επίπεδο μεταφοράς είναι υπεύθυνο για την επικοινωνία των δεδομένων που λαμβάνονται από το επίπεδο εφαρμογής του υπολογιστή αφετηρίας μέχρι το αντίστοιχο του προορισμού. Πρόκειται για μια επικοινωνία από \emph{άκρο σε άκρο} (end to end). Το επίπεδο μεταφοράς δεν ενδιαφέρεται για το γεγονός ότι στην πραγματικότητα τα δεδομένα του προωθούνται μέσα από ένα πλήθος άλλων κόμβων με τη βοήθεια του πρωτοκόλλου IP (στο επίπεδο διαδικτύου). Όσο αφορά το επίπεδο μεταφοράς, η σύνδεση είναι μια ευθεία γραμμή μεταξύ αφετηρίας και προορισμού. 

Υπάρχουν γενικά δύο τρόποι να ξεκινήσει μια επικοινωνία στο επίπεδο μεταφοράς:

\begin{itemize}
\item Ο κόμβος στην αφετηρία μπορεί να ξεκινήσει με την \emph{εγκατάσταση  σύνδεσης}: Θα επικοινωνήσει για αυτό το σκοπό με τον προορισμό και η μετάδοση των δεδομένων θα αρχίσει αφού καθοριστούν πρώτα οι παράμετροι της επικοινωνίας. Το πρόγραμμα στον υπολογιστή αφετηρίας επικοινωνεί και συνομιλεί με ένα παρόμοιο πρόγραμμα στον υπολογιστή προορισμού. Οι πληροφορίες της σύνδεσης αποθηκεύονται στις επικεφαλίδες του μηνύματος και στα μηνύματα ελέγχου.
\item Εναλλακτικά, ο κόμβος στην αφετηρία μπορεί απλά να ξεκινήσει να στέλνει δεδομένα προς τον προορισμό \emph{χωρίς να γίνει από πριν εγκατάσταση σύνδεσης}. 
\end{itemize}

Και στις δυο περιπτώσεις, τα δεδομένα που παράγονται στο επίπεδο μεταφοράς προωθούνται στην πραγματικότητα μέσα από το επίπεδο Διαδικτύου αφού ενθυλακωθούν μέσα σε πακέτα IP. Στη μεταφορά αυτή χρησιμοποιούνται πολλοί ενδιάμεσοι κόμβοι που εκτελούν υπηρεσίες δρομολόγησης (όπως είδαμε στο προηγούμενο κεφάλαιο).

Συνοπτικά, οι λειτουργίες που αναλαμβάνει το επίπεδο μεταφοράς είναι η εγκατάσταση και ο τερματισμός των συνδέσεων και ο έλεγχος ροής της πληροφορίας ώστε μια γρήγορη μηχανή να μην υπερφορτώνει μια αργή, καθώς και η επιβεβαίωση ότι η πληροφορία έφτασε πράγματι στον προορισμό της.

Τα δύο βασικά πρωτόκολλα στο επίπεδο μεταφοράς είναι το \emph{TCP,Transmission Control Protocol} και το \emph{UDP, User Datagram Protocol}. Το πρωτόκολλο TCP είναι \emph{προσανατολισμένο στη σύνδεση (connection oriented)} ενώ το UDP \emph{χωρίς σύνδεση (connectionless)}.

Τα πρωτόκολλα που είναι \emph{προσανατολισμένα στη σύνδεση} πριν ξεκινήσουν τη μετάδοση των δεδομένων εγκαθιστούν μια σύνδεση από άκρο σε άκρο μεταξύ των κόμβων για να εξασφαλίσουν μια διαδρομή (νοητή σύνδεση) για τη μετάδοση των πακέτων. Όλα τα πακέτα μεταδίδονται μέσα από τήν ίδια νοητή σύνδεση. Αφού ξεκινήσει η μετάδοση, εξασφαλίζουν ότι τα δεδομένα θα φτάσουν στον παραλήπτη χωρίς σφάλματα. (Σημείωση: εδώ το βιβλίο χρησιμοποιεί τον όρο ``πακέτα'' με την ευρεία έννοια: στο επίπεδο μεταφοράς η μονάδα δεδομένων του TCP είναι το τμήμα -- segment)

Αντίθετα στα πρωτόκολλα \emph{χωρίς σύνδεση} η μετάδοση δεδομένων ξεκινά άμεσα χωρίς να έχει προηγηθεί επικοινωνία με τον παραλήπτη. Τα δεδομένα μεταδίδονται σε \emph{αυτοδύναμα πακέτα} (datagrams) χωρίς την εγκατάσταση νοητής σύνδεσης. Τα πρωτόκολλα αυτά θεωρούνται αναξιόπιστα καθώς δεν εξασφαλίζουν ότι τα δεδομένα θα φτάσουν στο προορισμό τους.

Η πληροφορία που μεταφέρεται από άκρο σε άκρο στο επίπεδο μεταφοράς,  οργανώνεται σε ακολουθία από ομάδες δεδομένων που ονομάζονται datagrams. Κάθε datagram μετράται σε octets δηλ. οκτάδες δεδομένων ή bytes και αντιμετωπίζεται απολύτως ανεξάρτητα από το δίκτυο.

\begin{inthebox}
\textbf{Τι εννοεί το σχολικό βιβλίο στην παραπάνω παράγραφο;} Είναι κακογραμμένη και κατά βάση λάθος.  Είναι λάθος να μιλάμε για datagrams στο επίπεδο μεταφοράς αν δεν μιλάμε για το UDP. Το TCP έχει \emph{τμήματα (segments)} και όχι \emph{datagrams}. Στο επίπεδο μεταφοράς, τα τμήματα δεν έχουν την παραμικρή ιδέα για το ενδιάμεσο δίκτυο: η επικοινωνία για αυτά είναι μια ευθεία γραμμή. Στο επίπεδο Διαδικτύου, τα segments προφανώς ενθυλακώνονται σε IP datagrams και το καθένα αντιμετωπίζεται απολύτως ανεξάρτητα από το δίκτυο (μπορεί να ακολουθήσει διαφορετική διαδρομή, να διασπαστεί σε fragments κλπ).\\ 
\end{inthebox}

\textbf{Τι είναι τα octets;} Αναφέρονται σε ομάδες δεδομένων των 8 bit, αυτό δηλαδή που σήμερα όλοι αποκαλούμε bytes. Ωστόσο το byte δεν ήταν πάντοτε 8 bit όπως σήμερα (ως byte ορίζεται το ελάχιστο πλήθος bit που αποθηκεύεται σε μια διεύθυνση μνήμης και παλιότερα -- 1950 -- δεν αντιστοιχούσε σε 8 bit). Για το σκοπό αυτό και προκειμένου να μην υπάρχει σύγχυση πολλές φορές στα δίκτυα χρησιμοποιείται ο όρος οκτάδα ή octet. Στις μέρες μας βέβαια μπορείτε να λέτε ελεύθερα byte! 

 
