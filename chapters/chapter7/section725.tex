%
% section 7.2.5
%
\subsection{Διαχείριση Ασφάλειας}

\begin{inthebox}
Η \emph{Διαχείριση Ασφαλείας} σε ένα δίκτυο αναφέρεται στη διαχείριση πληροφοριών που σχετίζονται με την ομαλή λειτουργία του δικτύου, την παρακολούθηση και έλεγχο πρόσβασης σε τμήματα του ή όλο το δίκτυο και στην ασφάλεια των δεδομένων που διακινούνται και αποθηκεύονται σε αυτό.\\
\end{inthebox}

Για να ολοκληρωθεί το έργο της διαχείρισης ασφάλειας, πρέπει σε τακτά διαστήματα να συλλέγονται και να αναλύονται οι πληροφορίες που σχετίζονται με τους παραπάνω τομείς. Για το σκοπό αυτό χρησιμοποιούνται εργαλεία λογισμικού όπως:

\begin{itemize}
\item Πλατφόρμες συλλογής και ελέγχου δικτυακών δεδομένων (NMS Platforms)
\item Εργαλεία κρυπτογράφησης (cryptography tools)
\item Εργαλεία αυθεντικοποίησης (authentication tools) για τον έλεγχο πρόσβασης
\item Συστήματα ελέγχου εισβολέων (intrusion detection systems)
\item Τείχος προστασίας (firewall)
\item Εφαρμογή πολιτικών ασφαλείας (security policies)
\item Ημερολόγιο (αρχεία) καταγραφής (security logs) κ.α.
\end{itemize}

Καθένα από τα παραπάνω εργαλεία έχει διαφορετική εφαρμογή και στοχεύει να καλύψει επιμέρους ανάγκες ασφαλείας ενός δικτύου. Ένας διαχειριστής θα χρησιμοποιήσει περισσότερα από ένα εργαλεία για να εξασφαλίσει την ασφάλεια του δικτύου. Για το λόγο αυτό η διαχείριση ασφαλείας είναι μια αρκετά πολύπλοκη διαδικασία.

Για να είναι αποτελεσματική η διαχείριση ασφαλείας, θα πρέπει να προβλεφθούν οι πιθανές απειλές και τα σημεία κινδύνου ώστε να επιλεγούν τα σημεία που χρειάζονται μεγαλύτερη προσοχή και προστασία. Αφού γίνει αυτή η αναγνώριση, εγκαθίσταται και ρυθμίζεται το κατάλληλο λογισμικό. Μέσα από αυτό ο διαχειριστής παρακολουθεί και εντοπίζει πηγές κινδύνου και επιθέσεις στο δίκτυο στο συντομότερο χρονικό διάστημα. 