%
% section 7.3
%
\section{Πρότυπα Διαχείρισης}

Τα \emph{βασικά συστατικά ή οντότητες} από τις οποίες αποτελείται ένα τυπικό Σύστημα Διαχείρισης Δικτύου είναι:

\begin{itemize}
\item \textbf{Ο Διαχειριστής Δικτύου} (Manager Server)
\item \textbf{Ο Αντιπρόσωπος} (Agent)
\item \textbf{Η Βάση Πληροφοριών Διαχείρισης} (Management Information Base, MIB)
\end{itemize}

\begin{inthebox}
\textbf{Σημείωση:} Επειδή από τα παραπάνω, ίσως δεν είναι προφανές, θα πρέπει να ξεκαθαρίσουμε:

\textbf{Ο Διαχειριστής και ο Αντιπρόσωπος είναι προγράμματα που εκτελούνται σε μηχανήματα του δικτύου.}

Συγκεκριμένα, ο διαχειριστής (πρόγραμμα) χρησιμοποιείται από τον διαχειριστή (άνθρωπο) προκειμένου να λάβει πληροφορίες για την κατάσταση του δικτύου. Ο διαχειριστής (πρόγραμμα) εκτελείται σε ένα υπολογιστή που είναι επιφορτισμένος με τη διαχείριση του δικτύου και συχνά ονομάζεται Manager Server. 

Ο διαχειριστής (πρόγραμμα) συλλέγει δεδομένα επικοινωνώντας με τους αντιπροσώπους, που είναι αντίστοιχα προγράμματα που εκτελούνται σε κάθε τμήμα / συσκευή του δικτύου που διαθέτει δυνατότητα διαχείρισης. Προφανώς δεν είναι όλες οι συσκευές ή τα τμήματα του δικτύου κατάλληλα για την εκτέλεση προγραμμάτων αντιπροσώπου και άρα δεν είναι πάντα διαχειρίσιμα με κεντρικό τρόπο.  Δεν μπορούμε να έχουμε αντιπρόσωπο να εκτελείται σε ένα\ldots καλώδιο δικτύου. Επίσης φτηνές δικτυακές συσκευές δεν διαθέτουν συνήθως δυνατότητα διαχείρισης. Για παράδειγμα ένας φτηνός μεταγωγέας δικτύου (switch) όπως αυτός που έχετε πιθανόν στο σχολικό εργαστήριο δεν είναι διαχειριζόμενος. Για να διαθέτει μια συσκευή τέτοια δυνατότητα χρειάζεται να έχει επεξεργαστή, μνήμη και το κατάλληλο πρόγραμμα από τον κατασκευαστή της (firmware). Το πρόγραμμα αυτό δρα ως αντιπρόσωπος (agent) και επιτρέπει την ανταλλαγή πληροφοριών με το διαχειριστή (πρόγραμμα) μέσω πρωτοκόλλων διαχείρισης (π.χ. SNMP). Όταν φτιάχνουμε ένα δίκτυο μεσαίου ή μεγάλου μεγέθους προσπαθούμε να χρησιμοποιούμε δικτυακές συσκευές με δυνατότητα διαχείρισης όπου είναι δυνατόν. Μερικές φορές ωστόσο το κόστος μπορεί να είναι απαγορευτικό.\\
\end{inthebox}

Τα πιο γνωστά \emph{Πρότυπα Διαχείρισης Δικτύου (Network Management, NM)} είναι:

\begin{itemize}
\item Το \textbf{SNMP, Simple Network Management Protocol} του Διαδικτύου
\item To \textbf{CMIP, Common Management Information Protocol} του OSI
\end{itemize}
