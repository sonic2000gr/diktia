%
% section 7.2.1
%
\subsection{Παραμετροποίηση}

\begin{inthebox}
H \emph{Διαχείριση Παραμετροποίησης (Configuration Management, CM)} ασχολείται με την παρακολούθηση των παραμέτρων του δικτύου και των αλλαγών που συμβαίνουν σε αυτό.\\
\end{inthebox}

Τα προβλήματα που παρουσιάζονται σε ένα δίκτυο είναι συχνά αποτέλεσμα των αλλαγών που κάνει ο διαχειριστής στις ρυθμίσεις του. Αυτή η περιοχή διαχείρισης είναι πολύ σημαντική γιατί παρακολουθεί και καταγράφει όλες αυτές τις αλλαγές.

\begin{inthebox}
\textbf{Πότε γίνονται αλλαγές στις ρυθμίσεις;}

Αλλαγές στις ρυθμίσεις μπορεί να γίνουν όταν:

\begin{itemize}
\item Ο διαχειριστής του δικτύου προσθέτει ή αφαιρεί υπολογιστές ή δικτυακό υλικό.
\item Ο διαχειριστής προσθέτει ή αφαιρεί εφαρμογές (λογισμικό).
\item Ο διαχειριστής αλλάζει τις ρυθμίσεις μιας συσκευής την ώρα που αυτή είναι σε χρήση.
\item Γίνονται αυτόματες ενημερώσεις στο λογισμικό ή εγκατάσταση νέων εκδόσεων κλπ.
\item Γίνεται αναβάθμιση στα ενσωματωμένα προγράμματα (firmware) δικτυακών συσκευών (δρομολογητών, switch κλπ.)
\end{itemize}

\end{inthebox}

Μια σωστή διαχείριση παραμετροποίησης περιλαμβάνει την καταγραφή όλων των παραπάνω (και ακόμα περισσότερων) αλλαγών. Η καταγραφή μπορεί φυσικά να γίνεται χειροκίνητα χωρίς χρήση λογισμικού, αλλά σε οποιοδήποτε μεγάλο δίκτυο αυτό γενικά είναι χρονοβόρο και οδηγεί σε σφάλματα. Συνήθως χρησιμοποιείται κατάλληλο λογισμικό διαχείρισης παραμέτρων όπως τα CiscoWorks 2000 ή το Infosim.

Η διαχείριση παραμέτρων περιλαμβάνει τους παρακάτω στόχους:

\begin{itemize}
\item Τη συλλογή και αποθήκευση παραμέτρων των συσκευών του δικτύου, είτε τοπικά είτε από απόσταση (δεν είναι πάντα δυνατή η απομακρυσμένη διαχείριση μιας συσκευής: οι πιο απλές / φτηνές δικτυακές συσκευές δεν διαθέτουν περιβάλλον απομακρυσμένης διαχείρισης).
\item Την απλοποίηση της παραμετροποίησης των συσκευών.
\item Την παρακολούθηση αλλαγών που συμβαίνουν στις παραμέτρους.
\item Τη διαμόρφωση νοητών κυκλωμάτων μέσα από δίκτυα χωρίς μεταγωγή (non-switched networks). Πρόκειται για το λεγόμενο \emph{provisioning}: ο διαχειριστής βρίσκει διαδρομές και καθορίζει τα νοητά κυκλώματα που θα χρησιμοποιηθούν στην δικτυακή επικοινωνία.
\item Το σχεδιασμό και την πρόβλεψη μελλοντικών επεκτάσεων.
\end{itemize}

Η διαχείριση παραμέτρων υλικού και λογισμικού αποτελείται από πέντε ξεχωριστές δράσεις:

\begin{inthebox}
\textbf{Σημείωση:} Τα παρακάτω σημεία είναι μετάφραση στο βιβλίο σας από το αντίστοιχο άρθρο της \href{https://en.wikipedia.org/wiki/Configuration_management}{Wikipedia}. Προσπαθήσαμε εδώ να κάνουμε καλύτερη μετάφραση γιατί δυστυχώς στο σχολικό βιβλίο δεν βγαίνει νόημα\ldots\\
\end{inthebox}

\begin{itemize}
\item \textbf{Σχεδιασμός και Διαχείριση Παραμέτρων:} Πρόκειται για ένα επίσημο έγγραφο που περιγράφει τους τομείς και τις παραμέτρους με τις οποίες ασχολείται η διαχείριση και περιλαμβάνει μεταξύ άλλων:
\begin{itemize}
\item Το προσωπικό
\item Τις διαδικασίες εκπαίδευσης
\item Τα εργαλεία και τις διαδικασίες που πρέπει να ακολουθούνται
\item Τις μεθόδους ελέγχου κ.α.
\end{itemize}
\item \textbf{Ταυτοποίηση Παραμετροποίησης:} Ορίζει τις βασικές προδιαγραφές και παραμέτρους του δικτύου (baseline) με βάση τις οποίες γίνεται κατόπιν η παρακολούθηση των αλλαγών σε αυτό.
\item \textbf{Έλεγχος Παραμετροποίησης:} Περιλαμβάνει την αξιολόγηση όλων των προτάσεων για αλλαγές ή βελτιώσεις πάνω στο δίκτυο. Κατά τον έλεγχο κάποιες αλλαγές μπορεί να εγκρίνονται και άλλες να απορρίπτονται. 
\item \textbf{Κοστολόγηση Κατάστασης Παραμετροποίησης:} (Σημείωση: κανονικά δεν είναι κοστολόγηση, αλλά καταγραφή) Περιλαμβάνει την καταγραφή και αναφορά όλων των παραμέτρων του δικτύου (υλικού, λογισμικού, firmware κλπ) και των αποκλίσεων τους σε σχέση με τις αρχικές προδιαγραφές. 
\item \textbf{Επαλήθευση και Αξιολόγηση Παραμετροποίησης:} Πρόκειται για μια ανεξάρτητη έκθεση αξιολόγησης του υλικού και του λογισμικού του δικτύου προκειμένου να διαπιστωθεί αν τηρεί συγκεκριμένες προδιαγραφές που απαιτούνται από κανονισμούς (π.χ. για στρατιωτική χρήση κλπ.)
\end{itemize}