%
% section 7.2.3
%
\subsection{Διαχείριση Επιδόσεων}

\begin{inthebox}
Η \emph{Διαχείριση Επιδόσεων} (Performance Management ή Capacity Management) διασφαλίζει ότι η απόδοση του δικτύου βρίσκεται σε αποδεκτά επίπεδα, αυτά δηλαδή για τα οποία σχεδιάστηκε.\\
\end{inthebox}

Για το σκοπό αυτό, η διαχείριση επιδόσεων αξιολογεί κάποια μετρήσιμα χαρακτηριστικά απόδοσης όπως το χρόνο απόκρισης του δικτύου, την απώλεια πακέτων, τη χρήση των γραμμών επικοινωνίας, το βαθμό λαθών που συμβαίνουν κλπ. Οι πληροφορίες αυτές συλλέγονται σε ένα σύστημα διαχείρισης δικτύου (χρησιμοποιώντας πρωτόκολλα όπως το SNMP) με τους παρακάτω τρόπους:

\begin{itemize}
\item \textbf{Με συνεχή παρακολούθηση} και εκτίμηση της τρέχουσας κατάστασης από το διαχειριστή.
\item \textbf{Με ορισμό συναγερμών} στο σύστημα διαχείρισης δικτύου, το οποίο και θα μας ειδοποιήσει όταν τα επίπεδα απόδοσης μεταβληθούν σε σχέση με τα προκαθορισμένα και αποδεκτά.
\end{itemize}

Με σωστά σχεδιασμένη στρατηγική συλλογής και ανάλυσης δεδομένων απόδοσης, ο διαχειριστής μπορεί:

\begin{itemize}
\item Να πιστοποιήσει την αποτελεσματικότητα και αξιοπιστία του δικτύου.
\item Να προβλέψει τα προβλήματα πριν εμφανιστούν.
\item Να επανασχεδιάσει το δίκτυο για ακόμα καλύτερες επιδόσεις.
\item Να προετοιμάσει το δίκτυο για μελλοντικές βελτιώσεις / επεκτάσεις.
\end{itemize}

Για να εκτιμήσει καλύτερα την κατάσταση, ο διαχειριστής επιλέγει κάποιες παραμέτρους (πόρους) του δικτύου τους οποίους παρακολουθεί στενά.
