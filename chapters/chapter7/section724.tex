%
% section 7.2.4
%
\subsection{Διαχείριση Κόστους}

\begin{inthebox}
Η \emph{Διαχείριση Κόστους} (Accounting Management ή Billing Management) ασχολείται με την παρακολούθηση πληροφοριών κόστους που σχετίζονται με τη χρήση των πόρων ενός δικτύου.\\
\end{inthebox}

Βέβαια δεν παρέχουν όλα τα δίκτυα υπηρεσίες επί πληρωμή. Για παράδειγμα η τυπική χρήση ενός εταιρικού δικτύου δεν προκαλεί αύξηση του κόστους εκτός αν χρησιμοποιούνται υλικά που αναλώνονται (π.χ. εκτύπωση σε ένα δικτυακό εκτυπωτή) ή συνδέσεις δικτύου με ογκοχρέωση (π.χ. μισθωμένες γραμμές, δορυφορικές συνδέσεις). Στην περίπτωση αυτή προφανώς παρακολουθούνται μόνο οι συγκεκριμένες υπηρεσίες. Σε δίκτυα που δεν έχουν στόχο το κέρδος η έννοια αυτή αντικαθίσταται από την έννοια της Διοίκησης (Administration). 

Ανάλογα με την περίπτωση, ο σκοπός της διαχείρισης κόστους είναι:

Για επιχειρήσεις με στόχο το κέρδος:

\begin{itemize}
\item \textbf{Ο υπολογισμός του σωστού ποσού χρέωσης} που προκύπτει από τις επί πληρωμή υπηρεσίες στους αντίστοιχους χρήστες (ή ομάδες χρηστών, ή οργανισμούς)
\end{itemize}

Για επιχειρήσεις χωρίς στόχο το κέρδος:

\begin{itemize}
\item \textbf{Δημιουργία κοστολόγησης (ή σωστότερα: καταγραφής) της χρήσης των πόρων} του δικτύου ανά χρήστη ή ανά τμήμα για να προσδιοριστούν καλύτερα λειτουργίες όπως η λήψη αντιγράφων ασφαλείας ή ο συγχρονισμός των δεδομένων
\end{itemize}

Η διαχείριση κόστους επίσης καλείται να αναγνωρίσει και να εντοπίσει χρήστες ή ομάδες χρηστών του δικτύου που:

\begin{itemize}
\item \textbf{Παραβιάζουν τα δικαιώματα πρόσβασης} και επιβαρύνουν το δίκτυο με άσκοπες λειτουργίες
\item \textbf{Κάνουν μη αποτελεσματική χρήση του δικτύου}
\end{itemize}

Ο διαχειριστής είναι υπεύθυνος να αποφασίσει και να ορίσει τις παραμέτρους που θα παρακολουθούνται και θα καταγράφονται, τα χρονικά διαστήματα της καταγραφής καθώς και τον τρόπο υπολογισμού (αλγόριθμο) του κόστους. Αν δεν απαιτείται χρέωση, η συλλογή των δεδομένων θα χρησιμοποιηθεί για βελτιστοποίηση της απόδοσης.