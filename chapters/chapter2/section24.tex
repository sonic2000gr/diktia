%
% section 2.4
%
\setcounter{section}{3}
\section{Δίκτυα ETHERNET (10/100/1000Mbps)}

Το γνωστό μας πρότυπο Ethernet βασίζεται στις προδιαγραφές IEEE 802.3 αλλά καθώς εξελίσσεται έχουν δημιουργηθεί διάφορες παραλλαγές του. 

Η κωδικοποίηση των βασικών προτύπων γίνεται με την παρακάτω ονοματοδοσία:

\begin{center}
\textbf{Χ Base/Broadband Y}
\end{center}

Όπου:

\begin{itemize}
\item Το \textbf{X} είναι η ταχύτητα μετάδοσης των δεδομένων σε Mbps
\item Το \textbf{Base} υποδηλώνει μετάδοση \textbf{Βασικής Ζώνης} ένω το \textbf{Broad} μετάδοση \textbf{Ευρείας Ζώνης} (τύπος σηματοδοσίας)
\item Το \textbf{Y} υποδηλώνει το \textbf{μέγιστο μήκος} του \textbf{τμήματος (segment)}
\end{itemize}

\begin{inthebox}
\textbf{Τι είναι η σηματοδοσία βασικής και τι η ευρείας ζώνης;}

Όταν έχουμε να μεταδώσουμε ένα σήμα (είτε αναλογικό είτε ψηφιακό) μέσα από κάποιο φυσικό μέσο, υπάρχουν διάφορες πιθανότητες:

\begin{itemize}
\item Το φυσικό μέσο να μπορεί να μεταδώσει το σήμα όπως είναι αυτούσιο: Για παράδειγμα, το ηλεκτρικό σήμα (αναλογικό) μιας τηλεφωνικής συνομιλίας μπορεί να μεταδοθεί μέσα από τη γραμμή του τηλεφώνου χωρίς να χρειάζεται καμιά αλλαγή σε αυτό. Σε αυτή την περίπτωση έχουμε \textbf{μετάδοση βασικής ζώνης}.
\item Αντίθετα, αν θέλουμε για παράδειγμα να μεταδώσουμε μουσική μέσω ραδιοκυμάτων (σκεφτείτε το ραδιόφωνο), δεν μπορούμε να μεταδώσουμε απευθείας το ηλεκτρικό σήμα που αντιστοιχεί στον ήχο στον αέρα. Αντίθετα, πρέπει να χρησιμοποιήσουμε μια συχνότητα που είναι κατάλληλη για εκπομπή (σκεφτείτε όταν συντονίζετε το ραδιόφωνο σε ένα σταθμό) και πάνω σε αυτή να ``φορτώσουμε'' τα δεδομένα μας (τη μουσική). Αυτή είναι μια διαδικασία γνωστή ως διαμόρφωση: ουσιαστικά έχουμε μετατρέψει την αρχική μορφή των δεδομένων μας με τρόπο κατάλληλο για τη μετάδοση από το φυσικό μέσο. Αυτή είναι μια \textbf{μετάδοση ευρείας ζώνης}.
\end{itemize}

\textbf{Τι είναι το Τμήμα Ethernet; (Ethernet Segment)}

Σύμφωνα με τις προδιαγραφές του Ethernet, ένα τμήμα περιέχει το ομοαξονικό καλώδιο που χρησιμοποιείται ως φυσικό μέσο και τους υπολογιστές που βρίσκονται πάνω σε αυτό. Όταν φτάσουμε στο μέγιστο μήκος τμήματος, αν χρειάζεται να μεγαλώσουμε περισσότερο το δίκτυο χρησιμοποιούμε \emph{αναμεταδότες} (repeaters), ενώνοντας μεταξύ τους περισσότερα τμήματα. Υπάρχουν συγκεκριμένοι κανόνες και περιορισμοί στο πόσα τμήματα και αναμεταδότες μπορούν να συνδεθούν (δείτε τον \href{https://en.wikipedia.org/wiki/5-4-3_rule}{κανόνα 5-4-3 στην Wikipedia} αν ενδιαφέρεστε για λεπτομέρειες). Γιατί όμως δεν μπορούμε να μεγαλώσουμε απλά το καλώδιο;

Μπορούμε να σκεφτούμε δύο βασικούς λόγους:

\begin{itemize}
\item Ανάλογα με τον τύπο του καλωδίου, οι απώλειες μετά από μια απόσταση μπορεί να είναι τέτοιες που (σε συνδυασμό και με το θόρυβο) να οδηγούν σε αναξιόπιστη μετάδοση
\item Όταν γίνεται μια σύγκρουση στο δίκτυο, από μια απόσταση και μετά είναι πιθανόν οι σταθμοί να μη λαμβάνουν έγκαιρα το σήμα σύγκρουσης και να συνεχίζουν να μεταδίδουν
\end{itemize}
\end{inthebox}

Στον πίνακα \ref{t2-1} αναφέρονται βασικά πρότυπα του ΙΕΕΕ 802.3 και τα χαρακτηριστικά τους.


\begin{table}[!ht]
\begin{center}
\footnotesize
\begin{tabular}{|c|c|c|c|c|c|}
  \hline
    \multirow{2}{*}{}\textbf{Τύπος}&\textbf{Μέσο}&\textbf{Μέθοδος}&\textbf{Ρυθμός}&\textbf{Μέγιστο Μήκος} & \textbf{Τοπολογία}\\
    \textbf{Δικτύου}& \textbf{Μετάδοσης} & \textbf{Σηματοδοσίας} & \textbf{Δεδομένων} & \textbf{Τμήματος} & \\
  \hline
 \multirow{2}{*}{}\textbf{10Base5}& Ομοαξονικό & Βασικής Ζώνης & 10 Mbps & 500 m & Αρτηρίας\\
    & 50 Ohm Thick &&&&\\
  \hline
\multirow{3}{*}{}\textbf{10Base2}& Ομοαξονικό & Βασικής Ζώνης & 10 Mbps & 185 m & Αρτηρίας\\
    & 50 Ohm Thin &&&& \\
    & (RG-58) &&&&\\
  \hline
\multirow{3}{*}{}\textbf{1Base5}& Αθωράκιστο & Βασικής Ζώνης & 1 Mbps & 250 m & Αστέρα\\
    & συνεστραμμένο &&&& \\
    & (UTP)&&&& \\
  \hline
\multirow{3}{*}{}\textbf{10BaseT}& Αθωράκιστο & Βασικής Ζώνης & 10 Mbps & 100 m & Αστέρα\\
    & συνεστραμμένο &&&& \\
    & (UTP)&&&& \\
  \hline
\multirow{2}{*}{}\textbf{10Broad36}& Ομοαξονικό & Ευρείας Ζώνης & 10 Mbps & 3600 m & Αρτηρίας\\
    & 75 Ohm &&&& \\
  \hline
\end{tabular}
\caption{\textsl{Βασικά πρότυπα του IEEE 802.3 και τα χαρακτηριστικά τους}}
\label{t2-1}
\normalsize
\end{center}
\end{table}

Το πρότυπο Ethernet II είναι παρόμοιο με το 10Base5.

Εκτός από τις βασικές εκδόσεις Ethernet που αναφέρονται στον πίνακα, υπάρχουν και εκδόσεις για άλλα μέσα μετάδοσης, π.χ. για οπτική ίνα. Αυτά αναφέρονται με κωδικοποίηση 10Base-F (Fiber Ethernet).

Το πρότυπο \emph{10Base-F} βασίζεται στην προδιαγραφή \emph{FOIRL, Fiber Optic InterRe\-peater Link} η οποία δημιουργήθηκε για τη σύνδεση επαναληπτών (repeaters) μεταξύ τους με οπτικές ίνες. Συνήθως χρησιμοποιείται διπλή πολύτροπη οπτική ίνα 62.5/125 μm σε συνδυασμό με υπέρυθρο φως από LEDs. Η πιο συνηθισμένη παραλλαγή του προτύπου 10Base-F είναι η 10Base-FL που χρησιμοποιείται στη διασύνδεση κυρίως επαναληπτών σε απόσταση ως 2Km.

\parbox{\textwidth}{
\boxline
\textbf{Πως μεταφέρουν πληροφορία οι οπτικές ίνες; Τι είναι οι πολύτροπες και τι οι μονότροπες οπτικές ίνες;}\\

Οι οπτικές ίνες είναι ένα μέσο μετάδοσης φτιαγμένο να μεταδίδει φως αντί για ηλεκτρικό σήμα. Ουσιαστικά το φως εισέρχεται στο ένα άκρο της ίνας και μετά από διαδοχικές ανακλάσεις (φανταστείτε την ίνα σαν μια σειρά από μικρούς καθρέφτες) εξέρχεται από την άλλη μεριά. Οι οπτικές ίνες είναι κατάλληλες για μετάδοση ψηφιακών δεδομένων καθώς μπορούμε εύκολα να φτιάξουμε ένα κύκλωμα το οποίο να αναβοσβήνει με ταχύ ρυθμό το φώς στη μια άκρη, συμβολίζοντας έτσι τα δυαδικά 0 και 1.}

Οι πολύτροπες οπτικές ίνες είναι γενικά μεγαλύτερης διαμέτρου από τις μονότροπες. Ονομάζονται πολύτροπες επειδή το φως μπορεί να διαδοθεί μέσα σε αυτές με διάφορους τρόπους. Αυτό τις κάνει φτηνότερες στην κατασκευή αλλά περιορίζει το μέγιστο μήκος τους ή την ταχύτητα μετάδοσης που μπορούν να επιτύχουν. Στις πολύτροπες ίνες χρησιμοποιούμε συνήθως φως από LEDs (φωτοεκπέμπουσες διόδους) οι οποίες είναι αρκετά φτηνές. 

Οι μονότροπες ίνες είναι αρκετά μικρότερης διαμέτρου (τυπικά κάτω από 10 μικρόμετρα (μm)) και σε αυτές το φως μπορεί να διαδοθεί με ένα και μοναδικό τρόπο. Για το λόγο αυτό δεν μπορεί να χρησιμοποιηθεί κοινό φως από LED αλλά από LASER το οποίο έχει και μεγαλύτερο κόστος. Όμως οι ίνες αυτές προσφέρουν πολύ μεγαλύτερες ταχύτητες και αποστάσεις σε σχέση με τις πολύτροπες. \\
\boxline

Η οπτική ίνα χρησιμοποιείται συνήθως για να συνδέσουμε μεταξύ τους σημεία που απέχουν αρκετά (μέχρι 2Km) και όταν υπάρχει αυξημένος ηλεκτρομαγνητικός θόρυβος (π.χ. σε βιομηχανικό περιβάλλον και εγκαταστάσεις όπου υπάρχουν ηλεκτρικοί κινητήρες ή δημιουργούνται σπινθήρες κλπ). Η οπτική ίνα έχει τα μειονεκτήματα του αυξημένου κόστους και της δυσκολίας χειρισμού (δεν μπορούμε να τη λυγίσουμε όπως ένα καλώδιο, είναι αρκετά πιο δύσκολο να την κολλήσουμε όταν κόψει, απαιτεί ειδικά βύσματα κλπ.)

\subsection*{Ethernet Υψηλών Ταχυτήτων}

Στα πλαίσια των συνεχών βελτιώσεων και νέων εκδόσεων του Ethernet, δημιουργήθηκαν δύο νέα πρότυπα, το IEEE 802.3u (Fast Ethernet) και το IEEE 802.3z (Gigabit Ethernet). Το Fast Ethernet προσφέρει ταχύτητα 100Mbps ενώ το Gigabit, 1000 Mbps.

Για τη δημιουργία του Fast Ethernet, εκτός από το δεκαπλασιασμό της ταχύτητας από τα 10 στα 100Mbps, δόθηκε προσοχή ώστε να μη διαταραχθεί κατά το δυνατόν η υπάρχουσα καλωδιακή υποδομή. Καθώς το πιο συνηθισμένο (εμπορικά) μέχρι τότε πρότυπο Ethernet ήταν το 10Base-T, το Fast Ethernet σχεδιάσθηκε να χρησιμοποιεί επίσης τον ίδιο τύπο καλωδίου, δηλ. συνεστραμμένων ζευγών. Το απλό 10Base-T μπορεί να χρησιμοποιήσει καλώδιο κατηγορίας 3 (Cat3) ενώ το Fast Ethernet στην πλέον διαδεδομένη εκδοχή του (100Base-TX) χρειάζεται κατηγορίας 5 (Cat5). 

Τα επιμέρους πρότυπα του Fast Ethernet είναι:

\begin{itemize}
\item \textbf{100Base-TX:} Είναι το πιο διαδεδομένο στις μέρες μας πρότυπο. Χρησιμοποιεί καλώδιο UTP (αθωράκιστο) κατηγορίας 5 ή STP (θωρακισμένο). Η ταχύτητα φτάνει τα 100Mbps και η μετάδοση είναι Full Duplex δηλ. και προς τις δύο κατευθύνσεις (αμφίδρομη). Από τα τέσσερα (4) ζεύγη καλωδίων που διαθέτει το μέσο, χρησιμοποιούνται μόνο τα δύο (2). Το ένα ζεύγος χρησιμοποιείται για την αποστολή δεδομένων και το άλλο για τη λήψη. Για λόγους χρονισμού, στα δύο ζεύγη υπάρχει συνεχή μετάδοση συμβόλων, ακόμα και όταν δεν υπάρχουν δεδομένα προς μετάδοση (μεταδίδονται ειδικά σύμβολα σε αυτή την περίπτωση). Η απόσταση του τμήματος μπορεί να φτάσει τα 100 μέτρα. Τα δύο ζεύγη που δεν χρησιμοποιούνται συνήθως τερματίζονται (ώστε να μη λαμβάνουν θόρυβο από το περιβάλλον). 
\item \textbf{100Base-T4:} Σε αυτή την παραλλαγή του Ethernet, χρησιμοποιούνται και τα τέσσερα (4) ζεύγη καλωδίων. Αυτό αποτελεί μειονέκτημα σε παλιότερες εγκαταστάσεις όπου ενδεχομένως το εγκατεστημένο καλώδιο να μη διαθέτει πάνω από δύο ζεύγη. Το φυσικό μέσο είναι UTP κατηγορίας 3 και άνω. Η μέγιστη απόσταση τμήματος είναι και πάλι τα 100 μέτρα. Στα ζεύγη υπάρχει σήμα μόνο όταν υπάρχουν δεδομένα προς μετάδοση. Τα τρία ζεύγη χρησιμοποιούνται για μετάδοση δεδομένων, ενώ το τέταρτο για αναγνώριση (λήψη) των συγκρούσεων. Το πρότυπο 100Base-T4 δεν χρησιμοποιεί χωριστά κανάλια για την εκπομπή και τη λήψη, έτσι δεν είναι δυνατή η αμφίδρομη μετάδοση δεδομένων.
\item \textbf{100Base-FX:} Το πρότυπο αυτό χρησιμοποιεί διπλή πολύτροπη (62.5/125μm) ή μονότροπη οπτική ίνα. Το μήκος τμήματος φτάνει τα 412 μέτρα σε περίπτωση πολύτροπης ίνας και μονόδρομης (half duplex) επικοινωνίας και τα δύο (2) χιλιόμετρα σε Full duplex. Με χρήση μονότροπης ίνας, η μέγιστη απόσταση τμήματος μπορεί να φτάσει τα 25 χιλιόμετρα.
\end{itemize}

\subsection*{Gigabit Ethernet}

\begin{table}[!ht]
\begin{center}
\small
\begin{tabular}{|c|c|c|c|}
  \hline
    \multirow{2}{*}{}\textbf{Όνομα}&\textbf{Μέσο}&\textbf{Μέγιστο Μήκος} & \textbf{Χαρακτηριστικά}\\
    & \textbf{Μετάδοσης} & \textbf{Τμήματος} & \\
  \hline
 \multirow{1}{*}{}\textbf{1000Base-SX}& Οπτική Ίνα & 550 m & Πολύτροπη (50μm)\\
   \hline
 \multirow{1}{*}{}\textbf{1000Base-LX}& Οπτική Ίνα & 5000 m & Μονότροπη (9μm)\\
  \hline
 \multirow{4}{*}{}\textbf{1000Base-CX}& Χάλκινο καλώδιο & 25 m & STP\\
    & 2 ζεύγη STP &&\\
   & (Θωρακισμένο - &&\\
    & συνεστραμμένο) &&\\
  \hline
 \multirow{4}{*}{}\textbf{1000Base-T}& Χάλκινο καλώδιο & 100 m & Cat5 UTP\\
    & 4 ζεύγη STP &&\\
   & (Αθωράκιστο - &&\\
    & συνεστραμμένο) &&\\
  \hline
\end{tabular}
\normalsize
\caption{\textsl{Βασικά πρότυπα του IEEE 802.3z και τα χαρακτηριστικά τους}}
\label{t2-2}
\end{center}
\end{table}

Το πρότυπο IEEE 802.3z ή Gigabit Ethernet (πίνακας \ref{t2-2}), είναι το νεώτερο πρότυπο στην κατηγορία 802.3 και προσφέρει ταχύτητες 1000 Mbps (1 Gigabit). Χρησιμοποιεί καλώδιο τουλάχιστον κατηγορίας 5e (Cat5e από το enhanced, ενισχυμένο). Διαθέτει επίσης πρότυπα για λειτουργία μέσω οπτικής ίνας. Με χρήση πολύτροπης οπτικής ίνας 62.5μm στο πρότυπο 1000Base-SX το μέγιστο μήκος τμήματος φτάνει τα 275 μέτρα ενώ με ίνα 50μm φτάνει τα 550 μέτρα. Με μονότροπη ίνα των 9μm μπορεί να φτάσει τα 5Km.

Από όλες τις παραλλαγές του Gigabit Ethernet, η πιο συχνά χρησιμοποιούμενη είναι η 1000Base-T καθώς χρησιμοποιεί ουσιαστικά το ίδιο καλώδιο (Cat5) με το Fast Ethernet το οποίο υπάρχει ήδη σε πολλά εγκατεστημένα δίκτυα. Εκτός από το Gigabit Ethernet έχουν αναπτυχθεί και εκδόσεις με ταχύτητες 10 Gigabits, 40 Gigabits και 100 Gigabits (αναφέρονται ως 10Gb, 40Gb, 100Gb) ενώ βρίσκονται υπό ανάπτυξη δίκτυα ταχύτητας 400Gb.  
