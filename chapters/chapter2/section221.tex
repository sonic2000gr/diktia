%
% section 2.2.1
%
\subsection{Έλεγχος Λογικής Σύνδεσης (LLC -- IEEE 802.2)}

Το υπο-επίπεδο ελέγχου λογικής σύνδεσης είναι το ανώτερο του επιπέδου σύνδεσης δεδομένων και αποτελεί το συνδετικό κρίκο μεταξύ των ανώτερου επιπέδου (επίπεδο δικτύου) και του υπο-επιπέδου ελέγχου πρόσβασης στο μέσο. Κύριος σκοπός του LLC είναι η παροχή υπηρεσιών στο επίπεδο δικτύου το οποίο υποστηρίζεται από τα \emph{``Σημεία Πρόσβασης για Εξυπηρέτηση'' (SAPs -- Service Access Points)} που παρέχει το LLC. To LLC με τη σειρά του δέχεται υπηρεσίες από το υπο-επίπεδο MAC.

\begin{inthebox}
\textbf{Τι είναι τα Service Access Points;}

Μπορείτε να φανταστείτε τα SAPs σαν τα σημεία διεπαφής των επιπέδων. Για παράδειγμα, γνωρίζουμε ότι το TCP/IP δεν καθορίζει ακριβώς τις λειτουργίες του Επιπέδου Πρόσβασης Δικτύου. Όμως το επίπεδο Δικτύου, που βρίσκεται ακριβώς από πάνω, χρειάζεται να δώσει τις πληροφορίες του (πακέτα IP) σε αυτό το επίπεδο με κάποιο τυποποιημένο τρόπο (και ανεξάρτητα από το είδος του φυσικού δικτύου που ακολουθεί). Για το σκοπό αυτό, το υπο-επίπεδο LLC παρέχει μια τυποποιημένη διεπαφή για την επικοινωνία του με το επίπεδο δικτύου. Η διεπαφή αυτή παρέχει συγκεκριμένες υπηρεσίες και τρόπους επικοινωνίας ώστε να είναι δυνατή η επικοινωνία των επιπέδων. Διαφορετικά, για κάθε διαφορετικό τύπο δικτύου θα έπρεπε να ξαναγράφουμε κάποιες λειτουργίες που ανήκουν στο επίπεδο δικτύου.\\
\end{inthebox}

Το υπο-επίπεδο LLC μπορεί να παρέχει τις παρακάτω υπηρεσίες:

\begin{itemize}
\item \textbf{Υπηρεσία Χωρίς Επιβεβαίωση και Χωρίς Σύνδεση} -- Unacknowledged \\ Connectionless Service: Στην περίπτωση αυτή, ένα σταθμός εργασίας στέλνει πλαίσια (στο επίπεδο σύνδεσης δεδομένων, δεν έχουμε πλέον πακέτα αλλά πλαίσια, όπως θα δούμε στο Ethernet) στο σταθμό προορισμού χωρίς να περιμένει επιβεβαίωση λήψης. Πριν την έναρξη της επικοινωνίας δεν γίνεται καμιά συνεννόηση μεταξύ των σταθμών και ούτε υπάρχει διαδικασία τερματισμού σύνδεσης μετά το τέλος της μετάδοσης. Αν τα δεδομένα (π.χ. λόγω θορύβου) αλλοιωθούν ή χαθούν δεν γίνεται κάποια προσπάθεια επανάκτησης των αντίστοιχων πλαισίων. Καθώς δεν υπάρχει διαδικασία έναρξης σύνδεσης, η υπηρεσία αυτή παρέχει τη μικρότερη καθυστέρηση στην επικοινωνία και χρησιμοποιείται κυρίως για μεταδόσεις όπου το φυσικό μέσο παρουσιάζει μικρό ποσοστό λαθών και όπου η επανάκτηση λανθασμένων δεδομένων μπορεί να γίνει από ανώτερα επίπεδα.
\item \textbf{Υπηρεσία με Επιβεβαίωση Λήψης Χωρίς Σύνδεση} -- Acknowledged \\ Connectionless Service: Όπως και στην προηγούμενη υπηρεσία, δεν εγκαθίσταται σύνδεση πριν την έναρξη της μετάδοσης των δεδομένων. Για κάθε πλαίσιο όμως, ο παραλήπτης στέλνει ένα αντίστοιχο πλαίσιο επιβεβαίωσης λήψης. Η υπηρεσία αυτή εφαρμόζεται κυρίως σε συνδέσεις σημείου προς σημείου (point to point).
\item \textbf{Υπηρεσία με Σύνδεση} -- Connection Oriented Service: Πρόκειται για την πιο πολύπλοκη υπηρεσία που παρέχεται από το υπο-επίπεδο LLC. Για την έναρξη της επικοινωνίας, ο σταθμός εργασίας πρέπει να επικοινωνήσει με το σταθμό προορισμού και να εγκαταστήσει με αυτόν ένα \emph{νοητό κύκλωμα}. Η διαδικασία αυτή ουσιαστικά σημαίνει την εύρεση μιας συγκεκριμένης διαδρομής μέσω ενδιάμεσων κόμβων την οποία και θα ακολουθήσουν όλα τα πλαίσια μέχρι να φτάσουν στο προορισμό τους. Κατά τη διάρκεια της μετάδοσης γίνεται επίσης επιβεβαίωση λήψης κάθε πλαισίου καθώς και έλεγχος ροής των δεδομένων (ο έλεγχος ροής εξασφαλίζει ότι ο αποστολέας κάθε φορά θα στέλνει όσα δεδομένα είναι έτοιμος να δεχθεί ο παραλήπτης ώστε να αποφευχθούν φαινόμενα υπερχείλισης). Ο έλεγχος ροής αναφέρεται στο επίπεδο δικτύου.

Η διαδικασία εγκατάστασης νοητού κυκλώματος περιλαμβάνει τρία στάδια:
\begin{itemize}
\item \textbf{Την εγκατάσταση σύνδεσης:} Οι δύο σταθμοί δημιουργούν το νοητό κύκλωμα και ανταλλάσσουν κάποιες αρχικές τιμές για μεταβλητές και μετρητές που χρειάζονται για να παρακολουθήσουν τη μετάδοση των πλαισίων (παράδειγμα: συμφωνούν για πόση ώρα θα περιμένει ο αποστολέας μια επιβεβαίωση πλαισίου μέχρι να θεωρήσει ότι χάθηκε για να το μεταδώσει ξανά).
\item \textbf{Τη μεταφορά δεδομένων:} Μεταδίδονται τα πλαίσια και επιβεβαιώνεται η λήψη τους.
\item \textbf{Το τερματισμό της σύνδεσης:} Στη φάση αυτή ελευθερώνονται οι μεταβλητές και μετρητές, αποδεσμεύεται το φυσικό μέσο (τερματίζεται το νοητό κύκλωμα) και γενικά σταματά η χρήση οποιουδήποτε μέσου χρησιμοποιήθηκε για την επίτευξη της επικοινωνίας.
\end{itemize}
\end{itemize}