%
% section 3.1.3
%
\subsection{Σπατάλη Διευθύνσεων IP}

Έχουμε δει ότι ένα δίκτυο κλάσης C μπορεί να δεχθεί μέχρι 254 διαφορετικούς υπολογιστές. Φανταστείτε μια εταιρεία που χρειάζεται να συνδέσει 55 υπολογιστές στο Internet. Θα πρέπει να ζητήσει να της παραχωρηθεί μια περιοχή διευθύνσεων τάξης C. Όμως θα χρησιμοποιήσει μόνο 55 διευθύνσεις. Οι υπόλοιπες 199 θα παραμείνουν δεσμευμένες και ανεκμετάλλευτες. Αν βέβαια αργότερα η εταιρεία μεγαλώσει θα μπορέσει να χρησιμοποιήσει και τις υπόλοιπες διευθύνσεις που της έχουν παραχωρηθεί χωρίς να ζητήσει νέα περιοχή.

Τι γίνεται όμως αν μια εταιρεία χρειάζεται να συνδέσει 300 μηχανήματα; Αν ζητήσει μια περιοχή τάξης Β θα της αποδοθούν 65534 διευθύνσεις από τις οποίες θα χρησιμοποιήσει μόνο 300. \emph{Οι υπόλοιπες 65234 θα παραμείνουν δεσμευμένες και ανεκμετάλλευτες}. 

Ένας τέτοιος τρόπος διαμοιρασμού διευθύνσεων οδηγεί γρήγορα στην εξάντληση των διαθέσιμων διευθύνσεων IP (ειδικά της τάξης Β). Πέρα από τη \emph{σπατάλη και εξάντληση των διευθύνσεων} ο τρόπος αυτός οδηγεί και σε \emph{δυσχέρειες στη δρομολόγηση των πακέτων δεδομένων και στη διαχείριση των πινάκων δρομολόγησης}.

Σήμερα γενικά δεν χρησιμοποιούμε το σύστημα με τις τάξεις διευθύνσεων: αντί για αυτό κάθε διεύθυνση IP συνοδεύεται από ένα ακόμα αριθμό τη \emph{μάσκα δικτύου} με την οποία επιτυγχάνουμε τη λεγόμενη \emph{αταξική δρομολόγηση} (CIDR -- Classless Internet Domain Routing). Η αταξική δρομολόγηση περιγράφεται στα \href{https://www.ietf.org/rfc/rfc1519.txt}{RFC1519} και \href{https://www.ietf.org/rfc/rfc4632.txt}{RFC4632}.