%
% section 3.1.5
%
\subsection{Ειδικές Διευθύνσεις}
Εκτός από τις κανονικές διευθύνσεις IP που προορίζονται για συγκεκριμένους υπολογιστές (ονομάζονται διευθύνσεις \emph{αποκλειστικής διανομής ή unicast}) υπάρχουν και κάποιες ειδικές κατηγορίες διευθύνσεων.

\begin{itemize}
\item \textbf{Διεύθυνση Δικτύου:} Προσδιορίζει το δίκτυο στο οποίο ανήκει μια διεύθυνση. Για να βρούμε τη διεύθυνση δικτύου, χρειαζόμαστε μια διεύθυνση IP που να ανήκει στο δίκτυο και τη μάσκα. Η διεύθυνση δικτύου είναι ίδια με τη διεύθυνση IP στο κομμάτι που αντιστοιχεί στο τμήμα δικτύου (εκεί δηλαδή που τα αντίστοιχα ψηφία της μάσκας είναι άσοι) ενώ στο τμήμα του υπολογιστή έχει μηδενικά. Δείτε την προηγούμενη ενότητα για τον υπολογισμό της διεύθυνσης δικτύου καθώς και τον \href{http://www.freebsdworld.gr/diktia/subnetting-guide.pdf}{Οδηγό Ασκήσεων Υποδικτύωσης}.
\item \textbf{Διεύθυνση Εκπομπής:} Αφορά όλους τους υπολογιστές που βρίσκονται στο ίδιο δίκτυο. Όταν ένα πακέτο έχει διεύθυνση προορισμού τη διεύθυνση εκπομπής λαμβάνεται από όλους τους υπολογιστές που βρίσκονται στο ίδιο δίκτυο ή υποδίκτυο (όπως προσδιορίζεται από την αντίστοιχη μάσκα). Για να βρούμε τη διεύθυνση εκπομπής ξεκινάμε πάντα από τη \emph{διεύθυνση δικτύου} και \emph{θέτουμε σε ``1'' όλα τα ψηφία που ανήκουν στο τμήμα υπολογιστή}. Για παράδειγμα, στη διεύθυνση 192.168.1.18 με μάσκα 255.255.255.0, η διεύθυνση δικτύου είναι 192.168.1.0. Σε αυτή τη διεύθυνση θέτουμε σε ``1'' τα ψηφία της τελευταίας οκτάδας (που σύμφωνα με τη μάσκα αντιστοιχούν στο τμήμα υπολογιστή) και έχουμε το 192.168.0.255. \emph{Προσέξτε ότι σε μάσκες που δεν έχουν μόνο τις προφανείς τιμές 0 και 255 θα πρέπει να κάνουμε την επεξεργασία ψηφίο προς ψηφίο και προσεκτικά!} Δείτε επίσης και τον \href{http://www.freebsdworld.gr/diktia/subnetting-guide.pdf}{Οδηγό Ασκήσεων Υποδικτύωσης}.
\item \textbf{Διεύθυνση Πολυδιανομής:} Πρόκειται για διευθύνσεις κλάσης D οι οποίες προσδιορίζουν μια ομάδα υπολογιστών/κόμβων. Για παράδειγμα, οι δρομολογητές του υποδικτύου ακούνε στη διεύθυνση 224.0.0.2. Η υλοποίηση των τεχνικών πολυδιανομής περιγράφεται στο \href{https://www.ietf.org/rfc/rfc1112.txt}{RFC1112} και \href{http://www.iana.org/assignments/multicast-addresses/multicast-addresses.xhtml}{στην σελίδα της IANA} υπάρχει επίσημη λίστα αυτών των διευθύνσεων και η χρήση τους. 
\item \textbf{Διεύθυνση Επανατροφοδότησης ή Ανατροφοδότησης:} Γνωστή επίσης και με τις ονομασίες \emph{Loopback ή Local Loopback}. Πρόκειται για διευθύνσεις που ανήκουν στο δίκτυο 127.0.0.0/8 και συνήθως 127.0.0.0/32. Χρησιμοποιούνται για να κάνει ένας υπολογιστής δίκτυο με τον εαυτό του. Τα πακέτα που στέλνονται με προορισμό μια τέτοια διεύθυνση (συνήθως το 127.0.0.1 αλλά και οποιαδήποτε άλλη του 127.0.0.0/8) δεν φεύγουν ποτέ από τον υπολογιστή ούτε προωθούνται σε κάποια διεπαφή δικτύου. Απλά επιστρέφουν στον ίδιο υπολογιστή που τα έστειλε.

Η διεύθυνση αυτή λειτουργεί ακόμα και σε υπολογιστές που δεν είναι συνδεδεμένοι σε κάποιο δίκτυο, ακόμα και σε αυτούς που δεν διαθέτουν καν κάρτα δικτύου. Ο λόγος είναι ότι διάφορες υπηρεσίες μέσα σε ένα υπολογιστή μπορεί να επικοινωνούν μεταξύ τους μέσω δικτύου και είναι άσκοπο (και ενδεχομένως και επικίνδυνο από άποψη ασφάλειας) τα πακέτα τους να προωθούνται στο κανονικό φυσικό δίκτυο και πίσω στον ίδιο τον υπολογιστή. Με τη χρήση του Loopback Address μπορούν οι υπηρεσίες αυτές να λειτουργούν ακόμα και χωρίς να υπάρχει πραγματική δικτυακή σύνδεση.

\item \textbf{Διεύθυνση Limited Source 0.0.0.0/8:} Συναντάται μόνο ως \emph{διεύθυνση προέλευσης (source)} και δηλώνει πακέτα που προέρχονται από υπολογιστές του ίδιου δικτύου στο οποίο ανήκει και ο υπολογιστής που τα παραλαμβάνει. Αν τα πακέτα προέρχονται από διευθύνσεις τύπου 0.0.0.0/32, προέρχονται από τον ίδιο τον υπολογιστή που τα παραλαμβάνει.

\item \textbf{Διευθύνσεις Link Local 169.254.0.0/16:} Γνωστές και ως διευθύνσεις APIPA (Automatic Private IP Addressing). Για ευκολία στη διαχείριση, σε πολλά δίκτυα TCP/IP οι παράμετροι του δικτύου (διεύθυνση IP και άλλες ρυθμίσεις) δεν γίνονται χειροκίνητα σε κάθε μηχάνημα: υπάρχει ένας διακομιστής DHCP (θα δούμε σε επόμενη ενότητα) που στέλνει αυτές τις ρυθμίσεις αυτόματα σε κάθε μηχάνημα που συνδέεται. Σε περίπτωση που ένα μηχάνημα έχει ρυθμιστεί να λαμβάνει αυτόματα ρυθμίσεις αλλά ο διακομιστής DHCP δεν ανταποκρίνεται (π.χ. λόγω βλάβης), τότε θα πάρει μια τυχαία διεύθυνση από την περιοχή 169.254.0.0/16. Για αυτές τις διευθύνσεις θα βρείτε λεπτομέρειες στο \href{https://tools.ietf.org/html/rfc3927}{RFC3927}.
\item \textbf{Άλλες Ειδικές Διευθύνσεις IP:} Περιγράφονται στο \href{https://tools.ietf.org/html/rfc3330}{RFC3330 (Special-Use IPv4 Addresses)} 
\end{itemize}

Ο παρακάτω πίνακας συνοψίζει μερικές από τις παραπάνω κατηγορίες με παραδείγματα:

\begin{center}
\small
\begin{tabular}{|c|c|c|}
\hline
\textbf{Διεύθυνση} & \textbf{Ερμηνεία} & \textbf{Παράδειγμα} \\
\hline
\multirow{3}{*}{} Αποκλειστικής Διανομής & Προσδιορίζει ένα  & 192.168.1.3 \\
(unicast) & υπολογιστή (host)  & Ο υπολογιστής 192.168.1.3 \\
& (μια διασύνδεση) & \\
\hline
\multirow{3}{*}{} Πολυδιανομής & Προσδιορίζει ομάδα & 224.0.0.2 \\
(multicast) & (group) υπολογιστών & Δρομολογητές του \\
 & & δικτύου (κλάση D) \\
\hline
\multirow{3}{*}{} Εκπομπής ή Ακρόασης & προσδιορίζει όλους &  192.168.1.255 \\
broadcast & τους υπολογιστές ενός  & όλοι οι υπολογιστές του  \\
 &  δικτύου ή υποδικτύου & 192.168.1.0/24 \\
\hline
\multirow{2}{*}{} Επανατροφοδότησης & Αναφέρεται στον & 127.0.0.1 \\
Loopback & ίδιο υπολογιστή & το ίδιο το μηχάνημα \\
\hline
Link Local & Διεύθυνση APIPA & 169.254.54.21 \\
\hline 
\end{tabular}
\normalsize
\end{center}