%
% section 3.1
%
\section{Διευθυνσιοδότηση Internet Protocol Έκδοση 4\\ (IPv4)}
Το επίπεδο Δικτύου (Network) στο OSI ή το επίπεδο Διαδικτύου (Internet) στο TCP/IP:

\begin{itemize}
\item Παρέχει τη \emph{λογική διευθυνσιοδότηση} για όλα τα διασυνδεδεμένα μεταξύ τους δίκτυα.
\item Φροντίζει για την εύρεση της κατάλληλης \emph{διαδρομής} και παράδοσης του πακέτου στον τελικό κόμβο σε μια διαδικασία που ονομάζεται \emph{δρομολόγηση} (routing).
\end{itemize}

Στη διαδικασία της δρομολόγησης, το πακέτο μπορεί να διασπαστεί σε πολλά μικρότερα κομμάτια (fragments).  To βασικό πρωτόκολλο σε αυτό το επίπεδο είναι το \emph{Πρωτόκολλο Διαδικτύου ή IP (Internet Protocol)} το οποίο δημιουργεί \emph{αυτοδύναμα πακέτα} γνωστά ως \emph{datagrams (data+telegram)}. Καθώς είναι αυτοδύναμα, κάθε πακέτο μιας επικοινωνίας μπορεί να ακολουθήσει διαφορετική διαδρομή μέχρι τον παραλήπτη, με αποτέλεσμα να φτάσουν με διαφορετική σειρά. Κάποια επίσης μπορεί να μη παραδοθούν. Ωστόσο το επίπεδο Διαδικτύου είναι υπεύθυνο να τα ξαναβάλει στη σωστή σειρά και να ενημερώσει για πακέτα που δεν έφτασαν στον προορισμό τους ή αλλοιώθηκαν.    

Στο επίπεδο Διαδικτύου λειτουργεί επίσης το \emph{Πρωτόκολλο Μηνυμάτων Ελέγχου Διαδικτύου ή ICMP (Internet Control Message Protocol)} και το \emph{Πρωτόκολλο Διαχείρισης Ομάδων Διαδικτύου ή IGMP (Internet Group Management Protocol} τα οποία χρησιμοποιούνται κυρίως από δικτυακές συσκευές και λογισμικό συστημάτων και όχι τόσο από τελικούς χρήστες. Το ICMP χρησιμοποιείται για την αναφορά σφαλμάτων, μετάδοση ερωτημάτων και αναμετάδοση (relaying) διαγνωστικών μηνυμάτων. Εξαίρεση αποτελούν οι εντολές ping και traceroute.

\begin{inthebox}
\textbf{Τι είναι οι εντολές ping και traceroute;}

Οι δύο αυτές εντολές χρησιμοποιούν το πρωτόκολλο ICMP και μπορούν να χρησιμοποιηθούν σαν εργαλεία χρήστη (υπάρχουν εγκατεστημένες στα περισσότερα λειτουργικά συστήματα, μεταξύ άλλων στο Linux/UNIX και στα Windows). Η εντολή ping στέλνει πακέτα ICMP (για την ακρίβεια τύπου Echo Request) σε ένα παραλήπτη της επιλογής μας με σκοπό να δούμε αν υπάρχει δυνατότητα επικοινωνίας αποστολέα -- παραλήπτη. Ο παραλήπτης παραλαμβάνει το πακέτο και το επιστρέφει σε μας. Με την εκτέλεση της εντολής θα δούμε επίσης πόσα πακέτα στάλθηκαν, πόσα παραλήφθηκαν καθώς και το χρόνο που χρειάζεται για την απάντηση.

Για παράδειγμα, μπορούμε να ελέγξουμε αν ο υπολογιστής με διεύθυνση\\ 192.168.0.42 είναι διαθέσιμος στο τοπικό μας δίκτυο:

\small
\begin{verbatim}
[16:26:28][sonic@pegasus:~]$ ping 192.168.0.42
PING 192.168.0.42 (192.168.0.42): 56 data bytes
64 bytes from 192.168.0.42: icmp_seq=0 ttl=64 time=0.447 ms
64 bytes from 192.168.0.42: icmp_seq=1 ttl=64 time=0.461 ms
64 bytes from 192.168.0.42: icmp_seq=2 ttl=64 time=0.284 ms
^C
--- 192.168.0.42 ping statistics ---
3 packets transmitted, 3 packets received, 0.0% packet loss
round-trip min/avg/max/stddev = 0.284/0.397/0.461/0.080 ms
\end{verbatim}
\normalsize

Η εντολή \emph{traceroute} (σε κάποια λειτουργικά \emph{tracert}) χρησιμοποιείται για να μας δείξει από πόσους ενδιάμεσους κόμβους (hops) θα περάσουν τα πακέτα μας μέχρι να φτάσουν στον κόμβο προορισμού. Περισσότερες λεπτομέρειες θα δούμε σε επόμενη ενότητα. Για παράδειγμα, μπορούμε να δούμε πόσα hops χρειαζόμαστε από το τοπικό δίκτυο μέχρι τον υπολογιστή που παρέχει την δικτυακή τοποθεσία www.freebsdworld.gr μπορούμε να γράψουμε:

\footnotesize
\begin{verbatim}
[16:43:33][sonic@pegasus:~]$ traceroute www.freebsdworld.gr
traceroute to freebsdworld.gr (193.183.99.68), 64 hops max, 
40 byte packets

 1  router (192.168.0.250)  0.983 ms  0.614 ms  0.893 ms
 2  80.107.108.106 (80.107.108.106)  7.632 ms  7.701 ms  7.665 ms
 3  79.128.227.213 (79.128.227.213)  11.944 ms  11.210 ms
    79.128.226.245 (79.128.226.245)  11.790 ms
 4  gig5-0-5-gsr03.ath.OTEGlobe.gr (62.75.3.17)  11.356 ms
    62.75.3.157 (62.75.3.157)  11.321 ms  11.767 ms
 5  * * *
 6  212.162.19.109.ear2.Frankfurt1.Level3.net (212.162.19.109) 
    48.464 ms  48.998 ms  49.037 ms
 7  ae-122-3508.bar2.Milan1.Level3.net (4.69.159.126)  57.132 ms
    ae-121-3507.bar2.Milan1.Level3.net (4.69.159.122)  56.833 ms  
    57.746 ms
 8  212.73.241.130 (212.73.241.130)  57.867 ms  57.592 ms  64.055 ms
 9  217.171.38.134 (217.171.38.134)  57.439 ms  58.278 ms  58.169 ms
10  zms.gudgenet.com (193.183.99.68)  57.137 ms  57.998 ms  57.391 ms
\end{verbatim}
\normalsize
Μέχρι τον προορισμό περνάμε από 9 ενδιάμεσους κόμβους, ενώ βλέπουμε ότι πηγαίνουμε σε ένα προορισμό στο Μιλάνο μέσω\ldots Φρανκφούρτης!\\
\end{inthebox}

To IGMP χρησιμοποιείται για την ομαδοποίηση υπολογιστών και αποστολή ταυτόχρονα μηνυμάτων σε όλους τους υπολογιστές της ομάδας (streaming). Σε ένα υπολογιστή με TCP/IP, η υλοποίηση του ICMP είναι υποχρεωτική ενώ του IGMP προαιρετική.

\begin{figure}[!ht]
  \centering
  \includegraphics[width=0.95\textwidth]{images/chapter3/3-1}
  \caption {\textsl{Δίκτυο και Διαδίκτυο}}
  \label{3-1}
\end{figure}

Το πακέτο IP φτάνει ουσιαστικά χωρίς αλλαγές (αυτούσιο) από τον υπολογιστή του αποστολέα στον υπολογιστή προορισμού. Οι ενδιάμεσοι κόμβοι κάνουν μόνο μικρές αλλαγές στην επικεφαλίδα του πακέτου για διαχειριστικούς λόγους. Σε όλα τα ενδιάμεσα δίκτυα το πακέτο μπορεί να ενθυλακώνεται και να αποθυλακώνεται σε πλαίσια του 2ου επιπέδου, αλλά τα πλαίσια αυτά ισχύουν κάθε φορά μόνο για το συγκεκριμένο κομμάτι του δικτύου.

Για να το καταλάβουμε καλύτερα, ας δούμε το σχήμα \ref{3-1}. Ας θεωρήσουμε ότι το ΔΙΚΤΥΟ 1 και ΔΙΚΤΥΟ 2 είναι δύο δίκτυα Ethernet τα οποία ειναι συνδεδεμένα μεταξύ τους με το δρομολογητή που φαίνεται.  Στο ΔΙΚΤΥΟ 1, ο υπολογιστής συνδέεται μέσω ενός μεταγωγέα (Switch). Ο μεταγωγέας όπως ξέρουμε δουλεύει στο επίπεδο ζεύξης δεδομένων (μπορεί δηλ. να δει τις φυσικές διευθύνσεις για να αποφασίσει σε ποια θύρα θα στείλει ένα πλαίσιο) αλλά δεν γνωρίζει τις λογικές διευθύνσεις (διευθύνσεις IP). Ο υπολογιστής παράγει κάποια πακέτα IP τα οποία ενθυλακώνονται σε πλαίσια Ethernet και μεταδίδονται μέσω του μεταγωγέα στο δρομολογητή.

Στο δρομολογητή, τα πλαίσια αποθυλακώνονται και ενθυλακώνονται ξανά σε νέα πλαίσια αυτή τη φορά για το ΔΙΚΤΥΟ 2.  Το πακέτο IP σε όλη τη διαδρομή παρέμεινε ουσιαστικά το ίδιο, όμως αποθυλακώθηκε και ενθυλακώθηκε σε διαφορετικά πλαίσια.

Οι γραμμές μετάδοσης (γνωστές και ως ζεύξεις, κυκλώματα ή κανάλια) και οι συσκευές μεταγωγής -- δρομολογητές που επιτρέπουν σε δύο ακραίους υπολογιστές να επικοινωνήσουν μεταξύ τους ονομάζονται \emph{επικοινωνιακό υποδίκτυο}. Στα δίκτυα τεχνολογίας TCP/IP το επικοινωνιακό υποδίκτυο χρειάζεται να παρέχει λειτουργικότητα μέχρι το επίπεδο Διαδικτύου (ή το 3ο επίπεδο του OSI).

\begin{inthebox}
\textbf{Γιατί;}

Γιατί οι δρομολογητές που χρησιμοποιούνται κατά κύριο λόγο ως ενδιάμεσοι κόμβοι πρέπει να μπορούν να δουν τη λογική διεύθυνση προορισμού (διεύθυνση IP) προκειμένου να εκτελέσουν τη δρομολόγηση (να επιλέξουν δηλ. τη διαδρομή που θα στείλουν τα πακέτα). Η πληροφορία αυτή είναι διαθέσιμη μόνο στο επίπεδο Διαδικτύου, έτσι το επικοινωνιακό υποδίκτυο στο TCP/IP πρέπει να παρέχει λειτουργικότητα μέχρι τουλάχιστον αυτό το επίπεδο.\\
\end{inthebox}

Δύο ή περισσότερα ανεξάρτητα δίκτυα διασυνδεδεμένα μεταξύ τους ώστε να λειτουργούν ως ένα μεγάλο δίκτυο συνθέτουν ένα \emph{διαδίκτυο} (ή \emph{internet} με μικρό i).

Σε ένα δίκτυο υπολογιστών, για να μπορέσει η πληροφορία να φτάσει στον υπολογιστή προορισμού με τη μορφή πακέτου δεδομένων, θα πρέπει οι υπολογιστές να  \emph{προσδιορίζονται με μοναδικό τρόπο} με κάποιο σχήμα διευθυνσιοδότησης, όπως οι κατοικίες σε μια πόλη προσδιορίζονται από τον αριθμό, την οδό και το ταχυδρομικό κώδικα. Αυτή η διευθυνσιοδότηση γίνεται μέσω των λογικών διευθύνσεων IP που θα εξετάσουμε παρακάτω.