%
% section 3.1.4
%
\subsection{Μάσκα Δικτύου}

Για να αποφύγουμε τη σπατάλη διευθύνσεων, μαζί με τη διεύθυνση IP χρησιμοποιούμε ακόμα ένα αριθμό, μεγέθους 32 bit την \emph{μάσκα δικτύου (network mask)}. Η μάσκα διευκρινίζει ποια ψηφία της διεύθυνσης ανήκουν στο αναγνωριστικό δικτύου (network ID) και ποια στο αναγνωριστικό υπολογιστή (host id) μέσα στο συγκεκριμένο δίκτυο.

Με τον παραπάνω τρόπο:

\begin{itemize}
\item Όπου η μάσκα έχει άσους (1) το αντίστοιχο ψηφίο της διεύθυνσης IP ανήκει στο αναγνωριστικό δικτύου.
\item Όπου η μάσκα έχει μηδενικά (0) το αντίστοιχο ψηφίο της διεύθυνσης IP ανήκει στο αναγνωριστικό υπολογιστή.
\end{itemize}

Να σημειώσουμε εδώ ότι δεν είναι δυνατόν να \textbf{αναμειγνύονται μεταξύ τους οι άσοι και τα μηδενικά}. Η μάσκα ξεκινάει πάντα σαν μια σειρά από άσους και καταλήγει σε μηδενικά. Δεν μπορεί ένας άσος στα αριστερά του να έχει ένα μηδενικό, και ένα μηδέν στα δεξιά του ένα άσο.

Για παράδειγμα, η παρακάτω μάσκα δικτύου είναι σωστή:

\begin{center}
11111111 11111111 11111111 00000000
\end{center}

ενώ η επόμενη είναι \textbf{λάθος:}

\begin{center}
11111111 11110111 11111111 00000000
\end{center}

Η παραπάνω απαίτηση περιγράφεται στο έγγραφο \href{https://tools.ietf.org/html/rfc1812#section-2.2.5.1}{RFC1812 σελ.22}. Ένα παράδειγμα διεύθυνσης δικτύου με μάσκα φαίνεται παρακάτω:

\begin{center}
\small
\ttfamily
\begin{tabular}{|c|c|c|c|c|}
\hline
(δεκαδική μορφή) & 192. & 168. & 1. & 18. \\
\hline
Διεύθυνση IP: & 1100 0000 & 1010 1000 & 0000 0001 & 0001 0010 \\
\hline
Μάσκα: & 1111 1111 & 1111 1111 & 1111 1111 & 0000 0000\\
\hline
(δεκαδική μορφή) & 255. & 255. & 255. & 0 \\
\hline
\end{tabular}
\normalfont
\end{center}

H \emph{διεύθυνση δικτύου} στην οποία ανήκει ο υπολογιστής μπορεί να βρεθεί με την πράξη του \emph{λογικού KAI (AND)} μεταξύ των ψηφίων της διεύθυνσης IP και της μάσκας δικτύου. 

\begin{inthebox}
\textbf{Τι είναι το λογικό KAI;}

Είναι η λογική πράξη η οποία δίνει την τιμή ΑΛΗΘΗΣ (ή 1) μόνο όταν και οι δύο τιμές έχουν τιμή ΑΛΗΘΗΣ. Διαφορετικά, δίνει τιμή ΨΕΥΔΗΣ ή 0. Προσέξτε ότι οι λογικές πράξεις δεν είναι ίδιες με τις αντίστοιχες αριθμητικές! Ωστόσο μπορούμε να βρούμε το αποτέλεσμα του λογικού ΚΑΙ απλά πολλαπλασιάζοντας μεταξύ τους τα δυαδικά ψηφία.\\
\end{inthebox}

Στο προηγούμενο παράδειγμα μας, υπολογίζουμε την διεύθυνση δικτύου με τον τρόπο που φαίνεται  παρακάτω:

\begin{center}
\fontsize{9}{11}
\ttfamily
\begin{tabular}{|c|c|c|c|c|c|} 
\hline
\multirow{2}{*}{} Διεύθυνση & 1100 0000 & 1010 1000 & 0000 0001 & 0001 0010 & 192.168.1.18 \\
IP: & & & & & \\
\hline
Μάσκα: & 1111 1111 & 1111 1111 & 1111 1111 & 0000 0000 & 255.255.255.0 \\
\hline
\multirow{2}{*}{} Διεύθυνση & 1100 0000 & 1010 1000 & 0000 0001 & 0000 0000 & 192.168.1.0 \\
Δικτύου: & & & & & \\
\hline
\end{tabular}
\normalfont
\end{center}

\subsubsection*{Προκαθορισμένες Μάσκες Δικτύων Τάξης A,B,C}

Από τη περιγραφή που δώσαμε για τη μάσκα δικτύου, μπορούμε να υπολογίσουμε τις μάσκες για τις τρεις προκαθορισμένες κλάσεις δικτύων A,B,C που έχουμε δει. Γνωρίζουμε ότι για την κλάση Α, χρησιμοποιούνται 8 ψηφία στο τμήμα δικτύου, για την  κλάση B, 16 ψηφία και για την C, 24 ψηφία. Αυτός ο αριθμός των ψηφίων θα αντιπροσωπεύεται από άσους (1) στην αντίστοιχη μάσκα:

\begin{center}
\begin{tabular}{|c|c|c|c|c|c|c|c|}
    \rowcolor[gray]{0.95}
    \hline
    \multirow{2}{*}{} Τάξη & 1η Οκτάδα & \multicolumn{2}{|c|}{Δεκαδικό} & \multicolumn{2}{|c|}{Μάσκα} & Παρατηρήσεις \\ 
    \cline{3-6}
     \rowcolor[gray]{0.95}
    & & Από & Εώς & Δεκαδική & CIDR & \\
    \hline
    A & 0xxx xxxx & 0 & 127 & 255.0.0.0 & /8 & x: 0 ή 1\\
    \hline
    Β & 10xx xxxx & 128 & 191 & 255.255.0.0 & /16 & \\
    \hline
    C & 110x xxxx & 192 & 223 & 255.255.255.0 & /24 & \\
    \hline
 \end{tabular}
\end{center}

Όπως έχουμε πει, σε μια μάσκα δικτύου δεν μπορεί να αναμειγνύονται οι άσοι και τα μηδενικά. Αν μετρήσουμε όλους τους άσους στη μάσκα, μπορούμε αντί να τη γράψουμε στη κλασική της δεκαδική μορφή, να χρησιμοποιήσουμε τη μορφή CIDR: βάζουμε μια κάθετο (/) μετά τη διεύθυνση IP και τον αριθμό των άσων της μάσκας. Π.χ. σε ένα δίκτυο κλάσης C, η μάσκα είναι 255.255.255.0 δηλ. 24 άσοι και 8 μηδενικά. Αυτή η μάσκα σε μορφή CIDR γράφεται ως /24. Έτσι, για τη διεύθυνση 192.168.0.16 θα γράφαμε 192.168.0.16/24. Η μορφή CIDR είναι γνωστή επίσης ως \emph{πρόθεμα ή prefix}.