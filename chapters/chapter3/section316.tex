%
% section 3.1.6
%
\subsection{Υποδικτύωση}
Σε αρκετές περιπτώσεις είναι επιθυμητό να χωρίσουμε ένα δίκτυο σε μικρότερα \emph{υποδίκτυα}:

\begin{itemize}
\item \textbf{Οικονομία διευθύνσεων IP:} Όπως έχουμε δει, χρησιμοποιώντας τις τυποποιημένες κλάσεις δικτύων υπάρχει μεγάλη σπατάλη διευθύνσεων: Μια εταιρεία που χρειάζεται 2000 υπολογιστές δεν χρειάζεται τις πάνω από 65 χιλιάδες διευθύνσεις ενός δικτύου κλάσης Β. Οι διευθύνσεις που περισσεύουν πάνε χαμένες. Με την υποδικτύωση μπορούμε να μοιράσουμε ένα δίκτυο κλάσης Β σε 32 υποδίκτυα των 2048 διευθύνσεων και να τις κατανείμουμε σε πολλαπλές εταιρίες.
\item \textbf{Διαχειριστικοί λόγοι:} Μια εταιρεία μπορεί να διαθέτει λιγότερα από 254 μηχανήματα (ένα δίκτυο κλάσης C δηλαδή) αλλά σε διαφορετικούς χώρους ώστε να εξυπηρετούν διαφορετικούς σκοπούς. Π.χ. κάποια μηχανήματα στο λογιστήριο, άλλα στο γραφείο κίνησης, στην αποθήκη κλπ. Αντί όλα αυτά τα μηχανήματα να είναι συνδεδεμένα σε ένα δίκτυο, μπορούμε να το χωρίσουμε σε υποδίκτυα ένα για κάθε τμήμα. Το κάθε υποδίκτυο μπορεί αν χρειάζεται να επικοινωνεί με τα άλλα μέσω δρομολογητών (όπως θα δούμε παρακάτω) και ταυτόχρονα δεν αυξάνεται η άσκοπη κίνηση στο δίκτυο όταν οι υπολογιστές ενός τμήματος επικοινωνούν μόνο με άλλους στο ίδιο τμήμα.
\end{itemize}

Για να εκτελέσουμε την υποδικτύωση, θα ακολουθήσουμε τα παρακάτω βήματα:

\begin{itemize}
\item Το ζητούμενο μας μπορεί να είναι να χωρίσουμε το δίκτυο σε \emph{n πλήθος υποδικτύων} ή σε \emph{m πλήθος υπολογιστών}. Από τους αριθμούς αυτούς θα υπολογίσουμε μια \emph{νέα μάσκα δικτύου}
\item Θα υπολογίσουμε τις \emph{περιοχές διευθύνσεων} και τις διευθύνσεις υποδικτύου και εκπομπής για κάθε υποδίκτυο. Θα βρούμε την αρχική και τελική διεύθυνση IP που μπορούν να χρησιμοποιηθούν στους υπολογιστές που συνδέονται στο υποδίκτυο
\end{itemize}

\begin{inthebox}
\textbf{Σημείωση:} Όταν δίνουμε bits από το τμήμα υπολογιστή στο τμήμα δικτύου, έχουμε \textbf{υποδικτύωση}. Όταν δίνουμε bits από το τμήμα δικτύου στο τμήμα υπολογιστή έχουμε \textbf{υπερδικτύωση} (θα τη δούμε στην επόμενη ενότητα).

Για παραδείγματα και μεθοδολογία ασκήσεων δείτε και τον  \href{http://www.freebsdworld.gr/diktia/subnetting-guide.pdf}{Οδηγό Ασκήσεων Υποδικτύωσης}.\\
\end{inthebox}

\subsubsection*{Παράδειγμα 1}
Δίνεται η διεύθυνση δικτύου 192.168.3.0/24 (μάσκα 255.255.255.0).

\begin{itemize}
\item Να χωριστεί το δίκτυο σε τουλάχιστον έξι υποδίκτυα
\item Να δοθούν οι περιοχές διευθύνσεων κάθε υποδικτύου
\item Να δοθούν οι διευθύνσεις υποδικτύου και εκπομπής για κάθε υποδίκτυο
\item Πόσους υπολογιστές μπορεί να έχει το κάθε υποδίκτυο;
\end{itemize}

\begin{inthebox}
\textbf{Παρατηρήσεις για όλες τις ασκήσεις υποδικτύωσης:}

\begin{itemize}
\item Θα πρέπει να ξέρουμε να μετατρέπουμε δυαδικό σε δεκαδικό και αντίστροφα
\item Τα δεδομένα μπορεί να μας δίνονται σε δυαδικό ή (συνήθως) σε δεκαδικό. Για να κάνουμε υποδικτύωση θα πρέπει να μετατρέψουμε τα δεδομένα μας σε δυαδικό
\item Σε κάθε υποδίκτυο, η πρώτη διεύθυνση που βρίσκουμε είναι η \emph{διεύθυνση υποδικτύου} και η τελευταία η \emph{διεύθυνση εκπομπής} για το συγκεκριμένο υποδίκτυο
\item Σε κάθε υποδίκτυο ``χάνουμε'' δύο διευθύνσεις IP (την δικτύου και την εκπομπής). Π.χ. σε ένα υποδίκτυο με 16 διευθύνσεις, μόνο οι 14 είναι διαθέσιμες για IP υπολογιστών
\item Τα διαθέσιμα υποδίκτυα ή IP διευθύνσεις που προκύπτουν είναι πάντοτε δυνάμεις του δύο. Αν μας πουν να χωρίσουμε το δίκτυο σε 6 υποδίκτυα, θα το χωρίσουμε στην πραγματικότητα σε 8. Γιαυτό άλλωστε και η εκφώνηση περιέχει τη λέξη ``τουλάχιστον''
\end{itemize}
\end{inthebox}

Για να απαριθμήσουμε 6 υποδίκτυα, θα χρειαστούμε στην πραγματικότητα να δώσουμε τρία επιπλέον bit στο τμήμα δικτύου. Με δύο bit μπορούμε να απαριθμήσουμε μέχρι 4 υποδίκτυα (2\textsuperscript{2}=4) ενώ με 3 bit, 8 υποδίκτυα (2\textsuperscript{3}=8). Τελικά θα έχουμε 8 υποδίκτυα και όχι 6 (είναι σκόπιμο να θυμάστε ή να μπορείτε να υπολογίσετε γρήγορα τις δυνάμεις του 2 μέχρι το 2\textsuperscript{8}=256).

Τα τρία έξτρα bit που θα δώσουμε στο τμήμα δικτύου, θα φαίνονται στη νέα μάσκα δικτύου ως ``1''. Το πρώτο πράγμα που πρέπει να υπολογίσουμε είναι η νέα μάσκα δικτύου:

\begin{center}
\fontsize{11}{13}
\ttfamily
\begin{tabular}{|c|c|c|c|c|}
\hline
 \textbf{Διεύθυνση Δικτύου} & 192 & 168 & 3 & 0 \\ 
\hline
\multirow{2}{*}{} \textbf{Παλιά Μάσκα} & 255 & 255 & 255  & 0 \\ 
\cline{2-5} 
              \textbf{Δικτύου} & 11111111  & 11111111 & 11111111 & 00000000 \\ 
\hline
\multirow{2}{*}{} \textbf{Νέα Μάσκα} & 11111111 & 11111111 & 11111111 & \colorbox{red}{\color{white}{111}}00000  \\ 
\cline{2-5} 
             \textbf{Υποδικτύου} & 255 & 255 & 255 & 224 \\ 
\hline
\end{tabular}
\normalfont
\end{center}

Η νέα μάσκα δικτύου είναι 255.255.255.224 και σε μορφή CIDR, μπορούμε να γράψουμε 192.168.3.0/27. Οι διευθύνσεις των υπολογιστών είναι της μορφής

$<$Network\_ID$>$, $<$Subnet\_ID $>$, $<$Host\_ID$>$

με το τμήμα δικτύου (Network\_ID) να καταλαμβάνει 24 bits, το τμήμα υποδικτύου (Subnet\_ID) να καταλαμβάνει 3 bits και το τμήμα υπολογιστή (Host\_ID) να καταλαμβάνει τα υπόλοιπα 5 bits.

Είναι προφανές ότι με 5 bits στο τμήμα υπολογιστή μπορούμε να έχουμε σε κάθε υποδίκτυο συνολικά 2\textsuperscript{5}=32 διευθύνσεις IP. Από αυτές η πρώτη είναι πάντα η διεύθυνση δικτύου και η τελευταία η διεύθυνση εκπομπής του υποδικτύου. Άρα συνολικά σε κάθε υποδίκτυο θα έχουμε 30 διευθύνσεις IP που μπορούν να αποδοθούν σε υπολογιστές.

Σε σχέση με ένα δίκτυο κλάσης C, το οποίο μπορεί να έχει συνολικά 254 υπολογιστές, εδώ έχουμε λιγότερους γιατί χάνουμε δυο διευθύνσεις ανά υποδίκτυο. Έχουμε 8 υποδίκτυα με 30 διαθέσιμες IP, άρα συνολικά 240 διαθέσιμες IP. Η απώλεια αυτή είναι μικρή σε σχέση με τα οφέλη που αποκομίζουμε από την υποδικτύωση.

Για να γράψουμε τον πίνακα με τις περιοχές διευθύνσεων των υποδικτύων πιο εύκολα, παρατηρούμε τα παρακάτω:

\begin{itemize}
\item Στις οκτάδες που η μάσκα παραμένει 255 (11111111\textsubscript{2}) δεν υπάρχει κάποια αλλαγή σε σχέση με προηγουμένως. Οι υπολογισμοί μας στο συγκεκριμένο παράδειγμα αφορούν μόνο την τέταρτη οκτάδα
\item Υπάρχουν οκτώ υποδίκτυα που αντιστοιχούν στα αναγνωριστικά υποδικτύων από 0 ως 7 δηλ. από 000\textsubscript{2} ως 111\textsubscript{2}
\item Για κάθε υποδίκτυο θα πρέπει να πάρουμε όλες τις πιθανές τιμές στο τμήμα υπολογιστή δηλ. από 00000\textsubscript{2} ως 11111\textsubscript{2}
\item Σε κάθε υποδίκτυο, η πρώτη διεύθυνση είναι η διεύθυνση υποδικτύου και η τελευταία η εκπομπής. \textbf{Δεν χρειάζεται να κάνουμε έξτρα υπολογισμούς}
\item Όπως φαίνεται στο παράδειγμα, όταν η μάσκα δεν έχει τις ``προφανείς'' τιμές 0 ή 255, οι διευθύνσεις δικτύου/εκπομπής επίσης δεν είναι προφανείς και χρειάζονται υπολογισμό από το δυαδικό. Δώστε προσοχή στις πράξεις! 
\end{itemize}

Ο πίνακας είναι ο παρακάτω:

\begin{center}
\fontsize{10}{12}
\ttfamily
\begin{tabular}{|c|c|c|c|c|c|c|}
\hline
                Α/Α  & 1η οκτάδα  & 2η οκτάδα  & 3η οκτάδα  & \multicolumn{2}{c|}{ 4η οκτάδα }  & Διευθύνσεις \\ \hline

\multirow{2}{*}{0} & 11000000 & 10101000 & 00000011 & \multirow{2}{*}{\colorbox{red}{\color{white}{000}}}  & 00000 & 192.168.3.0 \\ \cline{2-4} \cline{6-7} 
                  & 11000000 & 10101000 & 00000011 &                    & 11111 & 192.168.3.31  \\ \hline

\multirow{2}{*}{1} & 11000000  & 10101000 & 00000011  & \multirow{2}{*}{\colorbox{red}{\color{white}{001}}}  & 00000  & 192.168.3.32 \\ \cline{2-4} \cline{6-7} 
                  & 11000000 & 10101000 & 00000011 &                    & 11111 & 192.168.3.63 \\ \hline
\multirow{2}{*}{2} & 11000000  & 10101000  & 00000011 & \multirow{2}{*}{\colorbox{red}{\color{white}{010}}}  & 00000 & 192.168.3.64 \\ \cline{2-4} \cline{6-7} 
                  & 11000000 & 10101000  & 00000011 &                    & 11111 & 192.168.3.95 \\ \hline
\multirow{2}{*}{3} & 11000000  & 10101000 & 00000011  & \multirow{2}{*}{\colorbox{red}{\color{white}{011}}}  & 00000 & 192.168.3.96 \\ \cline{2-4} 
\cline{6-7} 
                  & 11000000 & 10101000  & 00000011 &                    & 11111 & 192.168.3.127 \\ \hline
\multirow{2}{*}{4} & 11000000 & 10101000  & 00000011 & \multirow{2}{*}{\colorbox{red}{\color{white}{100}}}  & 00000 & 192.168.3.128 \\ \cline{2-4} \cline{6-7} 
                  & 11000000 & 10101000 & 00000011  &                    & 11111 & 192.168.3.159 \\ \hline
\multirow{2}{*}{5} & 11000000  & 10101000  & 00000011 & \multirow{2}{*}{\colorbox{red}{\color{white}{101}}}  & 00000 &  192.168.3.160 \\ \cline{2-4} 
\cline{6-7} 
                  & 11000000 & 10101000 & 00000011 &                    & 11111 &  192.168.3.191 \\ \hline
\multirow{2}{*}{6} & 11000000  & 10101000  & 00000011  & \multirow{2}{*}{\colorbox{red}{\color{white}{110}}}  & 00000 & 192.168.3.192 \\ \cline{2-4} \cline{6-7} 
                  & 11000000 & 10101000  & 00000011 &                    & 11111 & 192.168.3.223 \\ \hline
\multirow{2}{*}{7} & 11000000  & 10101000 & 00000011  & \multirow{2}{*}{\colorbox{red}{\color{white}{111}}}  & 00000  &  192.168.3.224 \\ \cline{2-4} \cline{6-7} 
                  & 11000000 & 10101000  & 00000011 &                    & 11111 & 192.168.3.255 \\ \hline
\end{tabular}
\normalfont
\end{center}

Στη στήλη Α/Α φαίνεται ο αύξοντας αριθμός του υποδικτύου (είναι ο αριθμός που αντιστοιχεί στο Subnet\_ID αλλά στο δεκαδικό). Οι υπολογιστές του κάθε υποδικτύου έχουν κοινές ολόκληρες τις τρεις πρώτες οκτάδες (το Network\_ID όπου η μάσκα έχει τιμή 255) και τα τρία πρώτα ψηφία της τέταρτης οκτάδας (που ανήκουν στο Subnet\_ID).

Η πρώτη διεύθυνση του πρώτου υποδικτύου είναι 192.168.3.0 και αποτελεί τη διεύθυνση δικτύου για το υποδίκτυο αυτό. Αντίστοιχα η τελευταία διεύθυνση,\\ 192.168.3.31, αποτελεί την διεύθυνση εκπομπής του υποδικτύου. Το ίδιο ισχύει και για τα υπόλοιπα υποδίκτυα. Σε κάθε υποδίκτυο υπάρχουν 32 διευθύνσεις, αλλά για υπολογιστές μπορούμε να χρησιμοποιήσουμε 30 (χάνουμε δύο εξαιτίας των διευθύνσεων δικτύου/εκπομπής).

\subsubsection*{Παράδειγμα 2}

Δίνεται η διεύθυνση δικτύου 192.168.17.0/24 (μάσκα 255.255.255.0). 

\begin{itemize}
\item Να χωριστεί το δίκτυο σε υποδίκτυα τουλάχιστον 50 υπολογιστών
\item Να δοθούν οι περιοχές διευθύνσεων του κάθε υποδικτύου
\item Να δοθούν οι διευθύνσεις υποδικτύου και εκπομπής κάθε υποδικτύου
\item Πόσα υποδίκτυα μπορεί να έχει το συγκεκριμένο δίκτυο;
\end{itemize}

Εργαζόμαστε με τον ίδιο τρόπο όπως προηγουμένως, μόνο που αυτή τη φορά ξεκινάμε με το πλήθος των υπολογιστών. Για να έχουμε 50 υπολογιστές, χρειαζόμαστε 6 bit στο τμήμα υπολογιστή. 2\textsuperscript{5}=32 οπότε τα 5 bit δεν επαρκούν, ενώ 2\textsuperscript{6}=64 οπότε στην πραγματικότητα το κάθε υποδίκτυο μας θα έχει 64 διευθύνσεις. Όπως και προηγουμένως, οι διευθύνσεις είναι πάντοτε δύναμη του 2, οπότε δεν μπορούμε να φτιάξουμε υποδίκτυα για 50 υπολογιστές αλλά για περισσότερους. 

Από αυτές τις 64 διευθύνσεις κάθε υποδικτύου η πρώτη θα είναι διεύθυνση δικτύου και η τελευταία εκπομπής, άρα για μηχανήματα κάθε υποδίκτυο θα διαθέτει 62 διευθύνσεις. Έχοντας δώσει 6 bit στο τμήμα υπολογιστή, μένουν 2 bit από την τέταρτη οκτάδα για το τμήμα υποδικτύου, οπότε θα έχουμε συνολικά 2\textsuperscript{2}=4 υποδίκτυα. Συνολικά μπορούμε σε όλα μαζί τα υποδίκτυα να συνδέσουμε 4 Χ 62 = 248 υπολογιστές, χάνουμε δηλ. 6 διευθύνσεις σε σχέση με την κανονική κλάση C. Σε σχέση με το προηγούμενο παράδειγμα, χάνουμε λιγότερες διευθύνσεις \emph{γιατί έχουμε λιγότερα υποδίκτυα}.

Θα υπολογίσουμε τώρα τη νέα μάσκα δικτύου:

\begin{center}
\fontsize{11}{13}
\ttfamily
\begin{tabular}{|c|c|c|c|c|}
\hline
 \textbf{Διεύθυνση Δικτύου} & 192 & 168 & 17 & 0 \\ 
\hline
\multirow{2}{*}{} \textbf{Παλιά Μάσκα} & 255 & 255 & 255  & 0 \\ 
\cline{2-5} 
              \textbf{Δικτύου} & 11111111  & 11111111 & 11111111 & 00000000 \\ 
\hline
\multirow{2}{*}{} \textbf{Νέα Μάσκα} & 11111111 & 11111111 & 11111111 & \colorbox{red}{\color{white}{11}}000000  \\ 
\cline{2-5} 
             \textbf{Υποδικτύου} & 255 & 255 & 255 & 192 \\ 
\hline
\end{tabular}
\normalfont
\end{center}

Η νέα μάσκα είναι 255.255.255.192 και σε μορφή CIDR μπορούμε να γράψουμε το δίκτυο 192.168.17.0/26. Οι διευθύνσεις των υπολογιστών είναι της μορφής 

$<$Network\_id$<$, $<$Subnet\_ID$>$, $<$Host\_ID$>$

όπου το Network\_ID είναι 24 bit, το Subnet\_ID είναι 2 bit και τα υπόλοιπα 6 bit ανήκουν στο Host\_ID. Ο πίνακας διευθύνσεων προκύπτει ακριβώς με τον ίδιο τρόπο του προηγούμενου παραδείγματος. Οι υπολογιστές του κάθε υποδικτύου έχουν κοινές τις τρεις πρωτές οκτάδες (όπου η μάσκα έχει τιμή 255) και τα δυο πρώτα ψηφία της τέταρτης οκτάδας.

\begin{center}
\fontsize{10}{12}
\ttfamily
\begin{tabular}{|c|c|c|c|c|c|c|}
\hline
                Α/Α  & 1η οκτάδα  & 2η οκτάδα  & 3η οκτάδα  & \multicolumn{2}{c|}{ 4η οκτάδα }  & Διευθύνσεις \\ \hline

\multirow{2}{*}{0} & 11000000 & 10101000 & 00010001 & \multirow{2}{*}{\colorbox{red}{\color{white}{00}}}  & 000000 & 192.168.17.0 \\ \cline{2-4} \cline{6-7} 
                  & 11000000 & 10101000 & 00010001 &                    & 111111 & 192.168.17.63  \\ \hline

\multirow{2}{*}{1} & 11000000  & 10101000 & 00010001  & \multirow{2}{*}{\colorbox{red}{\color{white}{01}}}  & 000000  & 192.168.17.64 \\ \cline{2-4} \cline{6-7} 
                  & 11000000 & 10101000 & 00010001 &                    & 111111 & 192.168.17.127 \\ \hline
\multirow{2}{*}{2} & 11000000  & 10101000  & 00010001 & \multirow{2}{*}{\colorbox{red}{\color{white}{10}}}  & 000000 & 192.168.17.128 \\ \cline{2-4} \cline{6-7} 
                  & 11000000 & 10101000  & 00010001 &                    & 111111 & 192.168.17.191 \\ \hline
\multirow{2}{*}{3} & 11000000  & 10101000 & 00010001  & \multirow{2}{*}{\colorbox{red}{\color{white}{11}}}  & 000000 & 192.168.17.192 \\ \cline{2-4} 
\cline{6-7} 
                  & 11000000 & 10101000  & 00010001 &                    & 111111 & 192.168.17.255 \\ \hline
\end{tabular}
\normalfont
\end{center}

Εννοείται ότι σε κάθε υποδίκτυο από τα παραπάνω η πρώτη διεύθυνση που υπολογίσαμε είναι η διεύθυνση δικτύου, ενώ η τελευταία η εκπομπής του συγκεκριμένου υποδικτύου.

\subsubsection*{Παράδειγμα 3}

Το παράδειγμα αυτό είναι αντίστοιχο με την ``Δραστηριότητα 3η'' του βιβλίου και αφορά υποδικτύωση σε δίκτυο κλάσης B.

Δίνεται η διεύθυνση δικτύου 134.55.0.0/16 (μάσκα 255.255.0.0, κλάσης B)

\begin{itemize}
\item Να χωριστεί το δίκτυο σε υποδίκτυα των 4000 υπολογιστών τουλάχιστον
\item Να υπολογιστεί η νέα μάσκα υποδικτύου
\item Να δοθούν οι περιοχές διευθύνσεων για τα τέσσερα πρώτα υποδίκτυα
\item Να δοθούν οι διευθύνσεις δικτύου και εκπομπής για τα τέσσερα πρώτα υποδίκτυα
\item Πόσα συνολικά υποδίκτυα έχει το συγκεκριμένο δίκτυο και πόσους υπολογιστές μπορούμε πραγματικά να συνδέσουμε σε κάθε υποδίκτυο;
\item Πόσες διευθύνσεις χάνουμε συνολικά λόγω της υποδικτύωσης σε σχέση με το κανονικό δίκτυο κλάσης B;  
\end{itemize}

Τη δεδομένη στιγμή, το δίκτυο διαθέτει 16 bit για το τμήμα δικτύου και 16 για το τμήμα υπολογιστή, δηλ. 65536 διευθύνσεις (65534 μηχανήματα). Για να έχουμε 4000 μηχανήματα, χρειαζόμαστε 12 bit γιατί 2\textsuperscript{12}=4096. Θα δώσουμε τα επιπλέον 4 bit από το τμήμα υπολογιστή, στο τμήμα δικτύου. Άρα:

\begin{itemize}
\item Με 4 bit στο τμήμα υποδικτύου μπορούμε να έχουμε συνολικά 2\textsuperscript{4}=16 υποδίκτυα
\item Με 12 bit στο τμήμα υπολογιστή, θα έχουμε συνολικά 4096 διευθύνσεις σε κάθε υποδίκτυο. Από αυτές θα μπορέσουμε να δώσουμε 4096-2=4094 σε μηχανήματα
\item Συνολικά χάνουμε 16 Χ 2 = 32 διευθύνσεις λόγω της υποδικτύωσης. Αντί για 65534 μηχανήματα, θα μπορούμε να συνδέσουμε 65504 (πολύ μικρή απώλεια σε σχέση με τα οφέλη της υποδικτύωσης)
\end{itemize}

Έχουμε απαντήσει ήδη στα δύο τελευταία ερωτήματα της άσκησης, κάνοντας μόνο τις απαραίτητες πράξεις με τα bit. Μπορούμε τώρα εύκολα να υπολογίσουμε τη νέα μάσκα υποδικτύου:

\begin{center}
\fontsize{11}{13}
\ttfamily
\begin{tabular}{|c|c|c|c|c|}
\hline
 \textbf{Διεύθυνση Δικτύου} & 134 & 55 & 0 & 0 \\ 
\hline
\multirow{2}{*}{} \textbf{Παλιά Μάσκα} & 255 & 255 & 0  & 0 \\ 
\cline{2-5} 
              \textbf{Δικτύου} & 11111111  & 11111111 & 00000000 & 00000000 \\ 
\hline
\multirow{2}{*}{} \textbf{Νέα Μάσκα} & 11111111 & 11111111 & \colorbox{red}{\color{white}{1111}}0000 & 00000000  \\ 
\cline{2-5} 
             \textbf{Υποδικτύου} & 255 & 255 & 240 & 0 \\ 
\hline
\end{tabular}
\normalfont
\end{center}

Έτσι η νέα μάσκα είναι 255.255.240.0 και σε μορφή CIDR μπορούμε να γράψουμε τη διεύθυνση δικτύου ως 134.55.0.0/20. Οι διευθύνσεις των υπολογιστών είναι πλέον της μορφής $<$Network\_ID$>$, $<$Subnet\_ID$>$, $<$Host\_ID$>$ όπου το Network\_ID είναι 16 bit, το Subnet\_ID είναι 4 bit και το Host\_ID 12 bit.

Οι περιοχές διευθύνσεων για τα τέσσερα πρώτα υποδίκτυα, μπορούν να υπολογιστούν εύκολα και φαίνονται στον παρακάτω πίνακα. Οι υπολογιστές του κάθε υποδικτύου έχουν κοινές τις δύο πρώτες οκτάδες (το Network\_ID, όπου η μάσκα έχει τιμή 255) και τα τέσσερα πρώτα ψηφία της τρίτης οκτάδας που αποτελούν το Subnet\_ID. 


\begin{center}
\fontsize{10}{12}
\ttfamily
\begin{tabular}{|c|c|c|c|c|c|c|}
\hline
    Α/Α              & 1η οκτάδα & 2η οκτάδα & \multicolumn{2}{c|}{3η οκτάδα} & 4η οκτάδα & Διευθύνσεις \\ \hline
\multirow{2}{*}{0} & 10000110  & 00110111 & \multirow{2}{*}{0000}  & 0000 & 00000000 & 134.55.0.0 \\ \cline{2-3} \cline{5-7} 
                  & 10000110 & 00110111 &                    & 1111 & 11111111 & 134.55.15.255 \\ \hline
\multirow{2}{*}{1} & 10000110  & 00110111 & \multirow{2}{*}{0001}  & 0000 & 00000000 & 134.55.16.0 \\ \cline{2-3} \cline{5-7} 
                  & 10000110 & 00110111 &                    & 1111 & 11111111 & 134.55.31.255 \\ \hline
\multirow{2}{*}{2} & 10000110 & 00110111 & \multirow{2}{*}{0010}  & 0000 & 00000000 & 134.55.32.0 \\ \cline{2-3} \cline{5-7} 
                  & 10000110  & 00110111 &                    & 1111 & 11111111 & 134.55.47.255 \\ \hline
\multirow{2}{*}{3} & 10000110  & 00110111 & \multirow{2}{*}{0011}  & 0000 & 00000000 & 134.55.48.0 \\ \cline{2-3} \cline{5-7} 
                  & 10000110 & 00110111 &                    & 1111 & 11111111 & 134.55.63.255 \\ \hline
\end{tabular}
\normalfont
\end{center}

Σε κάθε υποδίκτυο, η πρώτη διεύθυνση είναι η διεύθυνση δικτύου και η τελευταία η διεύθυνση εκπομπής. Έτσι π.χ. για το υποδίκτυο με Α/Α 0, η διεύθυνση δικτύου είναι 134.55.0.0 και η εκπομπής 134.55.15.255. Προσέξτε ότι οι 4096 διευθύνσεις προκύπτουν από συνδυασμό τιμών των δύο τελευταίων οκτάδων (δεν είναι απλά μια πρόσθεση όπως ίσως κάνατε στην κλάση C).